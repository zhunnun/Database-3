\documentclass[a4paper,12pt]{article}
\usepackage{graphicx}
\usepackage{color}
\usepackage[utf8]{inputenc}
\usepackage[bahasa]{babel}

\title{\Large{\textbf{Rangkuman Database}}\\Diajukan Untuk Memenuhi Salah Satu Tugas \\Mata Kuliah Basis Data I}
\author{}
\date{}
\begin{document}
\maketitle
\thispagestyle{empty}
\begin{center}
\includegraphics[width=5cm,height=5cm]{logo.png}\\
\hfill\break
\textbf{DOSEN PENGAMPU}\\
Syafrial Fachrie Pane, S.T., M.T.I.,EBDP\\
\hfill\break
\textbf{DISUSUN OLEH}\\
Muhammad Yaqdhan Taqy Ariana\\(1194027)\\Kelas 1A \\D4-Teknik Informatika\\ 
\vfill
\textbf{PROGRAM DIPLOMA IV TEKNIK INFORMATIKA}\\
\textbf{POLITEKNIK POS INDONESIA}\\
\textbf{2020}\\
\end{center}
\newpage
\setcounter{page}{1}
\section{Rangkuman Database}
	\subsection{Definisi \textit{Database}}
		\paragraph{} Basis data atau juga disebut \textit{database} merupakan pengumpulan data berupa bilangan, huruf, angka, simbol, file yang akan disimpan ke dalam suatu tempat dan saling berelasi.Konsep database sendiri berasal dari \underline{himpunan}.
		\paragraph{}Dari pernyataan berikut, \textit{database} berfungsi sebagai :\\
		\begin{itemize}
		\item Pengelompokan data untuk mempermudah identifikasi data, karena pengambilan data harus sesuai fakta dan berasal dari sumber yang relevan.
		\item Menghindari redudansi atau data yang duplikat karena \textit{database} mempunyai kapasitas tempat yang terbatas.
		\item Memecahkan masalah penyimpanan data yang tradisional.
		\end{itemize}
\subsection{Jenis-jenis \textit{Database}}
		\paragraph{}\textit{Database} memiliki jenis-jenis seperti:\\
		\begin{itemize}
		\item SQL\textit{(Structured Query Language)}\\
		Contohnya :
		\begin{enumerate}
		\item MySql
		\item Oracle
		\item Microsoft Access
		\end{enumerate}
		\item NoSQL\\
		Contohnya :
		\begin{enumerate}
		\item MongoDB
		\item FireBird
		\end{enumerate}
		\end{itemize}
\newpage
\subsection{Perbandingan \textit{Database} Konvensional dan Tersistem}
\paragraph{}\textit{Database} konvensional merupakan sebuah metode untuk menyimpan data, biasanya berbentuk \textit{hardfile} berupa barang fisik seperti kertas dan lain lain. dan di simpan dalam sebuah tempat biasanya berupa kabinet. Kelebihan nya ialah ketika akan mencari data, keaslian nya akan terbukti karena memang data yang disimpan merupakan bukti konkrit. Namun memiliki kekurangan jika terjadi kerusakan yang diakibatkan oleh kesalahan manusia maupun bencana alam, data tersebut akan hilang.Kekurangan lainnya ialah akan memakan tempat yang banyak jika pangkalan data hanya satu tempat, karena data akan terus bertambah dan tempat penyimpanan akan sedikit.
\paragraph{}Sedangkan \textit{Database} tersistem mengutamakan media digital berupa internet dan server, keuntungan menggunakan basis data ini ialah lebih fleksibel, mudah di\textit{backup}, dan mudah di akses. Sedangkan kekurangan nya ialah apabila pangkalan data tersebut mengalami galat. kemungkinan nya akan sulit mem-\textit{backup}data tersebut dengan kata lain harus mengakses pangkalan data tersebut secara lokal.
\end{document}

