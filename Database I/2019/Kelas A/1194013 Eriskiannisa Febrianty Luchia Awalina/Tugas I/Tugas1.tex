\documentclass{article}
\usepackage[utf8]{inputenc}

\title{Tugas Rangkuman Database}
\author{}
\date{}

\usepackage{natbib}
\usepackage{graphicx}

\begin{document}
\maketitle
\begin{center}
\includegraphics{download.png}
\end{center}
\hfill\break

\begin{center}
\begin{tabular}{c c l}

    Nama & : & Eriskiannisa febrianty Luchia Awalina\\
    NPM & : & 1194013\\
    Kelas & : & 1A\\
    Prodi & : & D4 Teknik Informatika\\

\end{tabular}
\end{center}

\newpage
\section{Pengertian Database}
Database merupakan kumpulan data yang berisi informasi yang dapat diisi dengan angka, huruf, simbol, dan gambar.Dikumpulkan dalam satu tempat, terstruktur dan tersusun berdasarkan fakta yang dapat dipertanggung jawabkan dan data tersebut saling berelasi.


\section{Fungsi Database}
\begin{itemize}
\item Mempermudah dalam identifikasi data yang berisi informasi dan dapat dicari secara cepat.
\item Memecahkan masalah penyimpanan data yang memakan banyak ruang.
\item Menghindari terjadinya data ganda.
\end{itemize}

\section{Tabel Database}
Tabel dalam database menampung informasi serta digunakan untuk menyusun dan mengklasifikasikan data. Database tersebut terstruktur agar tidak ada data yang duplikat atau ganda. 

\section{Penyimpanan Database}
Memori yang menjalankan sistem database adalah DBMS (Database Management System). data yang sudah dikumpulkan disimpan didalam DBMS dan akan ternomalisasi agar tidak ada data yang redudansi.

\section{Jenis-jenis Database}
\begin{itemize}
    \item MySQL
    \item Oracle
    \item Microsoft Acces
    \item Maria DB
    \item Microsoft SQL Server
    \item Virtual Foxpro
\end{itemize}

\section{Perkembangan Database}
Database merupakan tempat penyimpanan yang sudah ada sejak dulu. Dulu database disimpan secara manual. Dalam kehidupan sehari-hari database ibarat dompet yang tersusun untuk menyimpan uang dan berbagai macam kartu yang digunakan. Dulu database masih berupa kertas atau dokumen yang berisi informasi yang disimpan di dalam ruang penyimpanan yang sudah disediakan. Jika dokumen yang berisi infromasi tersebut semakin banyak dan bertambah seiring waktu, maka ruang penyimpanan yang diperlukan pun akan semakin besar. Sekarang database menggunakan sistem yang dapat menyimpan banyak informasi dalam satu tempat dan tidak memakan banyak ruang penyimpanan. 


\bibliographystyle{plain}
\end{document}
