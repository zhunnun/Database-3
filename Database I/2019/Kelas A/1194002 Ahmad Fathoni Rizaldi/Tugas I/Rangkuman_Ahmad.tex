\documentclass[11pt]{book}
\title{Rangkuman Basis Data I}
\author{Ahmad Fathoni Rizaldi}
\date{2019-02-27}

\begin{document}
\begin{flushleft}
Nama		: Ahmad Fathoni Rizaldi
\\
NPM			: 1194002
\\
Kelas		: D4 TI 1A
\\
Mata Kuliah	: Basis Data I
\\
Tugas		: Rangkuman Basis Data I
\end{flushleft}
\paragraph{1. Pengertian database}
\paragraph{	Database merupakan kumpulan data berupa nomor, tulisan, maupun gambar yang tersimpan dalam komputer secara sistematis sehingga dapat diakses melalui program komputer agar dapat memperoleh suatu informasi dari dalam basis data tersebut. Database ditemukan oleh seseorang yang  bernama Charles Bachman yang menciptakan generasi pertama DBMS(Database Management System).} 
\paragraph{	Basis data sendiri memiliki cara untuk mengaksesnya, seperti menggunakan aplikasi yaitu MySQL, Microsoft Access, Oracle, dll. tujuan dibuat database itu bertujuan agar tidak terjadi redudansi atau data yang terbentuk secara ganda pada saat proses normalisasi. karena database yang baik itu memiliki data-data yang berbeda dan fleksibel.}
\paragraph{2. Implementasi Basis Data Dalam Kehidupan Sehari-hari}
\paragraph{	Tidak hanya dalam komputer, basis data ini juga bisa diterapkan dalam kehidupan sehari-hari contohnya adalah lemari, dompet, laci, rak, dll. benda tersebut mampu menampung sebuah benda dan mengkategorikannya menjadi lebih terstruktur dan rapi. karena apabila benda diletakan secara terstruktur dan rapi akan lebih mudah untuk kita mengambil suatu benda dari tempat tersebut dan benda tersebut tidak tercecer kemana-mana.}
\end{document}