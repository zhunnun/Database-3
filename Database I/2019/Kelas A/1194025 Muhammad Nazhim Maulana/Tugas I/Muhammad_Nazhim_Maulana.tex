\documentclass[a4paper, 12pt]{article}
\usepackage{color}
\usepackage[utf8]{inputenc}
\usepackage[bahasa]{babel}
\usepackage{graphicx}
\title{\textbf{Rangkuman}\linebreak \\ \textbf{Basis Data}}
\date{}
\begin{document}
\maketitle
\begin{center}
\includegraphics[width=5 cm, height=5 cm]{logo.jpg}\linebreak
\end{center}
\begin{center}
\textbf{Di susun}\linebreak
\textbf{ Oleh :} \linebreak
\end{center}
\vspace{0.5 cm}
\begin{flushleft}
Nama : Muhammad Nazhim Maulana \\
NPM : 1194025 \\
Kelas : D4 TI 1A\\
\end{flushleft}
\newline
\newline
\newline
\begin{center}
Untuk memenuhi Tugas Basis Data \\
Dosen Pengampu: Syafrial Fachri Pane, ST., MTI., EBDP. \linebreak
\newline

\newline
Program Study Teknik Informatika \\
Politeknik Pos Indonesia \linebreak
Tahun Ajaran 2019/2020 \linebreak
\end{center}
\begin{center}
\title{\textbf{Rangkuman Materi Database}\linebreak}
\end{center}
\begin{flushleft}
\title{\textbf{A. Pengertian Database}\linebreak}
\paragraph{}Data atau lebih dikenal dengan Database merupakan kumpulan dari beberapa informasi (baik dalam bentuk bilangan, angka, huruf, maupun dalam bentuk teks). Database pada dasarnya berdasarkan dengan materi himpunan pada pelajaran matematika. Tujuan merancang sebuah database adalah untuk mencegah adanya data yang ganda atau duplicate.
\paragraph{} Kumpulan informasi dalam database itu disimpan ke dalam sebuah penyimpanan bernama DBMS (Database Manajemen System) yang berupa software. Terdapat beberapa software DBMS, diantaranya :
\end{flushleft}
\begin{flushleft}
\paragraph{} 
\begin{itemize}
\item[•] MySql
\item[•] Oracle
\item[•] Microsoft SQL Server
\end{itemize}
\end{flushleft}
\newline
\begin{flushleft}
\title{\textbf{B. Contoh Penerapan Database Di kehidupan Sehari-Hari}\linebreak}
\paragraph{}Database atau basis data sudah di kenal di kalangan masyarakat sejak dulu, dalam kehidupan sehari-hari saja kita memanfaatkan database untuk menyusun barang yang kita miliki. Sebagai contoh saja dompet, dalam dompet kita dapat menyimpan dan mengolompokkan beberapa barang, yaitu ada bagian untuk menyimpan uang, ada bagian untuk menyimpan foto dan ada juga bagian yang digunakan untuk menyimpan kartu identitas diri.
\paragraph{}Selain dompet, contoh lain benda yang menganut prinsip database ada tas yang sering kita gunakan untuk berangkat sekolah. Pada tas kita bisa mengelompokkan barang berdasarkan ukurannya. Barang yang berukuran besar dimasukkan ke dalam bagian yang punya ruangan besar dan batang yang ukurannya kecil dimasukkan ke dalam bagian tas yang memiliki ruangan yang kecil.
\end{flushleft}
\newline
\end{document}