\documentclass[a4paper]{article}
\usepackage{graphicx}

%opening
\title{Rangkuman Basis Data I}

\begin{document}
	\maketitle
	
\begin{center}
	\includegraphics[width=5cm,height=5cm]{logo-poltekpos.png}\\
\end{center}
\begin{center}
		\begin{tabular}{c c l}
		
		
		Nama & : & Dzul Jalali Wal Ikram \\
		Npm & : & 1194011 \\
		Kelas & : & 1-A \\
		Prodi & : & D4 Teknik Informatika \\
	\end{tabular}
\end{center}
\newpage

	\section{Pembahasan}
	\subsection{Definisi}
	\paragraph{}
	Data base adalah beberapa informasi yang berupa angka, huruf, gambar yang saling berelasi, disimpan dalam suatu tempat dimana data tersebut akan terstruktur secara sistematis.
	\subsection{Tujuan}
	\paragraph{}
	Tujuan basis data adalah untuk menghindari data ganda atau redudansi.
	\subsection{Jenis-jenis Database}
		\subsection*{SQL}
	\paragraph{}
	contoh :
\begin{itemize}

	\item MySQL
	\item Oracle
\end{itemize}

\subsection*{Non SQL}
\paragraph{}
contoh :
	\begin{itemize}
	\item MongoDB
	\item OrientDB
	\item CouchDB
\end{itemize}

	\subsection{Konsep relasi cloud  dan DataBase}
	\paragraph{}
	Cara kerja relasi, informasi yang terdapat pada cloud akan dikirim ke database untuk disimpan, Kemudian data yang terdapat pada database dikirim kembali ke Cloud untuk di backup.
	
	\subsection{Perbedaan Database Manual dengan Database system}
	\paragraph{}
	\textbf{Database manual} merupakan pengumpulan data yang tidak efektif karena masih menggunakan cara primitif, berkas berkas penting diarsipkan dan disimpan dalam brankas yang membuat pencarian data atau file memakan banyak waktu, dan juga jika terjadi suatu kecelakaan seperti kebakaran data data penting tidak akan bisa dibackup.
	\paragraph{}
	\textbf{Database system} merupakan pengumpulan data yang sangat efektif, karena tidak hanya cepat dalam pencarian data atau file \textbf{database system} juga bisa dibackup jika terjadi kecelakaan Karena terhubung dengan internet dan cloud. Sehingga jika data di suatu device hilang bisa di download lewat cloud storage.
	

\end{document}
