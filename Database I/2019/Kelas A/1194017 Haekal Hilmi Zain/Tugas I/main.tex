\documentclass{article}
\usepackage[utf8]{inputenc}

\title{Rangkuman Basis Data 1}
\author{Haekal Hilmi Zain }
\date{27 February 2020}

\usepackage{natbib}
\usepackage{graphicx}

\begin{document}

\maketitle
\begin{center}
\includegraphics{Poltekpos.jpg}
\end{center}

\newpage
\section{Basis Data}
\paragraph{} Basis data / data base adalah kumpulan data yg disimpan dalam suatu tempat secara sistematis dan dapat diolah untuk menghasilkan informasi. Yang berupa angka huruf dan gambar. contoh penerapan database dalam kehidupan sehari hari yaitu dompet yang berisi informasi seperti uang, ktp dan atm. 


\subsection{Tujuan Basis Data }
\paragraph{} Tujuan dibuatnya database adalah mempermudah dan mengefisien waktu dalam mencari data dan menghindari data ganda atau redudansi.

\subsection{Jenis - Jenis Basis Data }
\begin{itemize}
    \item MySQL
    \item Oracle
    \item dBase
    \item Visual Foxpro
\end{itemize}

\subsection{Backup Database }
\paragraph{} Data yang terdapat pada cloud dipindahkan ke database untuk disimpan. Data yang berada di database dikirim kembali ke cloud untuk di backup.

\subsection{Database manual dan system }
\paragraph{}Database manual pengumpulan data yang kurang efektif karena masih menggunakan cara primitif. Data atau berkas berkas disimpan dalam suatu brangkas/ruangan yang membuat pencarian data memakan banyak waktu dan memiliki resiko data bisa rusak.

\paragraph{}Database system kumpulan data yang efektif. Karena waktu pencarian data dengan cepat dan juga gampang untuk di backup agar hal hal yang tidak diinginkan tidak terjadi

\end{document}
