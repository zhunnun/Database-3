\documentclass[10pt]{article}
\title{DATABASE}
\author{}

\usepackage [bahasa]{babel}
\begin{document}
\maketitle
\begin{center}
\begin{tabular}{ll}
Nama 				&: \textbf {M. Rizky}\tabularnewline\\
NPM       			&: \textbf {1194021}\tabularnewline\\
Kelas 				&: \textbf {1A}\tabularnewline\\
Materi 				&: \textbf {Rangkuman LaTeX}\tabularnewline\\
Dosen Pengampu		&: \textbf {Syafrial Fachri Pane, ST., MTI., EBDP.}\tabularnewline\\
\end{tabular}
\newline
\newline
\newline
\newline
\newline
\newline
\newline
\textbf {POLITEKNIK POS INDONESIA} \linebreak
TEKNIK INFORMATIKA\linebreak
2019/2020
\end{center}
\newpage
\begin{abstract}
\section{Database}
\subparagraph{} Sebelum kita mengenal apa itu database, alangkah baiknya kita mengenal terlebih dahulu arti arti kata kunci dari database itu sendiri. \emph {Database} biasa kita kenal dengan kumpulan data yang disimpan dalam suatu wadah/software databse itu sendiri yaitu seperti : mongoDB, Oracle, Firebase, Mysql, dll. Arti dari kata kumpulan disini adalah beberapa angka, huruf, bilangan, dll, yang disatukan.
\subparagraph{} Kemudian, data - data yang dikumpulkan tadi akan di olah dan dimanipulasi lagi dengan system yang terstruktur dan sistematis, sehingga nantinya akan menghasilkan sebuah informasi yang biasa dinamakan \emph {System Informasi}. Tujuan dari adanya database sendiri selain untuk menyimpan data, yaitu untuk terhindar dari data ganda/redudansi yang bisa membuat data - data kita menjadi tidak teratur dan terarah.
\subparagraph{} Data - data yang kita ambil dan kita simpan kedalam database harus berupa data real dan ada fisiknya dan juga dalam pengambilan datanya harus dengan menganalisa, mengumpulkan, dan diuji apakah data sudah benar atau belum.
\subparagraph{} Jadi, kesimpulannya \emph {Database} adalah kumpulan dari data - data yang kita kumpulkan kemudian di simpan kedalam wadah berupa software database yang kemudian nantinya akan di olah dan dimanipulasi lagi dengan system yang terstruktur dan sistematis sehingga nantinya akan menghasilkan sebuah system informasi.
 
\end{abstract}
\end{document}
