\documentclass[12pt, times new roman, a4paper]{article}
\usepackage{pspicture}
\usepackage{graphicx}
\usepackage{hyperref}
\usepackage[latin1]{inputenc}

\title{Tugas Rangkuman Basis Data}
\author{Salsa Thalia Putri Nur Rochmani}
\date{27 Februari 2020}

\begin{document}
\maketitle

Basis Data adalah suatu kumpulan data atau informasi yang dapat berupa huruf, angka, simbol, atau gambar yang dikumpulkan dalam satu tempat secara tersusun dan sistematik dimana datanya tersusun berdasarkan fakta agar bisa dipertanggung jawabkan, dan data tersebut harus saling berelasi.

\begin{itemize}
  \item Sebelum berkembang, Basis data masih berupa buku besar, kuitansi, dan kumpulan data yang berelasi dengan dunia bisnis atau keuangan yang digunakan oleh suatu perusahaan dan sistemnya pun masih tradisional. selain itu, cara ini memerlukan banyak tempat penyimpanan dan memakan banyak biaya. tetapi pada masa sekarang ini Basis Data sudah berbasis Teknologi Informasi dan Telekomunikasi sehingga Basis Data bisa lebih efektif dan efisien dari pada menggunakan data-data primitif dengan cara yang lama.
  \item Basis Data tercipta dari penyederhanaan cara lama yang tradisional agar  menjadi lebih efektif dan efisien. 
  \item Basis Data berfungsi untuk mengelompokkan data dan mempermudah identifikasi data. Basis Data juga menjadi solusi untuk mengatasi masalah penyimpanan data konvensional yang memerlukan ruang yang besar dan memakan biaya banyak.
  \item Data yang dikumpulkan dalam Basis Data terstruktur dan konsisten sehingga terhindar dari data ganda.
  
\end{itemize}



\end{document}
