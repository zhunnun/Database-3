\documentclass[a4paper, 12 pt]{article}
\author{}
\usepackage{color}
\usepackage[bahasa]{babel}
\usepackage{graphicx}
\title{\textbf{Rangkuman Basis Data}\linebreak}
\date{}
\begin{document}
	
	\maketitle
	\begin{center}
		\includegraphics[width=6cm]{logo.png}
	\end{center}
	\vspace{0.5 cm}
	
	\begin{center}
		Alwizain Almas Trigreisian \\
		1194004 \\
		D4TI 1A \linebreak
		\newline
		\newline
		\newline
		Untuk Memenuhi Tugas Basis Data I Semester 2 \\
		Dosen Pengampu: Syafrial Fachri Pane, ST., MTI., EBDP. \linebreak
		\newline
		\newline
		Program Studi D4 Teknik Informatika \\
		Politeknik Pos Indonesia\\
		2019/2020\\
	\end{center}
	
	\newpage
	\begin{flushleft}
		\title{\textbf{Rangkuman Basis Data}}\linebreak
	\end{flushleft}

	\par Database atau Basis Data merupakan kumpulan beberapa informasi atau data berupa 	gambar, angka, huruf, symbol yang saling berelasi dan berdasarkan fakta. Data-data 			atau informasi dalam database tersebut dibuat tabel yang memiliki attribut saling 			berelasi antara tabel satu dengan yang lainnya. Jadi konsep pada database ini 				seperti pada irisan himpunan di Matematika yang data-datanya saling berelasi. Adapun 	tempat untuk penyimpanan data atau yang disebut dengan DBMS (Database Management 			System) dalam database tersebut antara lain Oracle, Mysql, dan Firebird. Database 			juga memiliki tujuan, yaitu antara lain:
		\par 1. Untuk pengelompokan data
		\par 2. Mencegah data yang ganda
		\par 3. Memudahkan akses ke data
		\par 4. Menyederhanakan dari berbagai jenis dan banyaknya data\\ \linebreak

	\par Pada zaman dahulu sudah terdapat database, namun penerapannya masih manual. 			Contohnya seperti penyimpanan berkas data prajurit yang harus disimpan berupa 				lembaran-lembaran dan dikelompokkan berdasarkan jenis prajurit karena pada zaman 			dahulu belum adanya database pada sistem komputer. Namun sekarang database sudah 			menggunakan sistem dan sangat sederhana sehingga dapat mempermudah pengguna untuk 			pengelompokan data. Data-data pada database tidak diperbolehkan hanya berupa opini 			saja, namun juga harus memiliki bukti yang bisa dipertanggungjawabkan. \linebreak

	\par Dalam pembuatan program atau aplikasi yang membutuhkan database, kita harus 			menganalisis kebutuhan terlebih dahulu yang nantinya kebutuhan-kebutuhan tersebut 			dimasukkan kedalam sistem database. Sehingga program atau aplikasi yang kita buat 			dapat terancang dengan baik. Banyak kasus diantara kita yang membuat sistem dahulu, 		baru membuat database kemudian. Hal tersebut kurang benar dalam penerapannya. 				Seharusnya kita membuat database terlebih dahulu lalu pembuatan sistem kemudian. 			Sehingga kebutuhan dapat dianalisa dan dapat terciptanya sistem yang baik.
\end{document}