\documentclass[12pt]{article}

\title{BASIS DATA 1}
\author{AYU LESTARI}
\date{27/02/2020}

\begin{document}
\maketitle
Pengertian Basis Data
Basis data atau database adalah kumpulan informasi berupa  hurup ,angka ,bilangan, text yang di simpan atau di tempatkan di dalam DBMS(Database management system)  seperti (mysql, mongoDB, oracle, sql serve ) secara otomatis, fakta, nyata, valid, dan konsisten. Database berawal dari himpunan, di dalam himpunan mempunyai  elemen atau atribut .  Pada zaman dahulu database sudah ada namun, database dulu belum mempunyai keamanan yang bagus dan tempat penyimpan data tidak fleksibal. Dan semakin berkembangnya zaman  sekarang database tempat penyimpanannya menjadi fleksibel. Tujuan database adalah kecepatan dalam mencari data, ddapat diolah secara efektif dan efesien  dan aman menyimpan data.
Perbedaan   basis data  pada zaman modern :


\begin{itemize}
  \item	Mudah untuk  mengakses data 
  \item Menghindari terjadinya duplikat data atau redudancy
  \item Tempat penyimpanan data  fleksibel dan mudah dibawa kemana-mana
  \item Mempunyai  keamanan bagus
  \item data bisa di backup di computer lain\\

\maketitle perbedaan   basis data pada zaman  tradisional :
  \item	susah untuk mengakses data 
  \item	mudah terjadi redundancy atau duplikat data
  \item tempat penyimpanan terbatas
  \item keamanan data masih rawan
\end{itemize}
\maketitle
contoh penerapan  database pada zaman dahulu ialah pedagang gorengan yang dimana di dalam gerobak itu dikelompokan berdasarkan jenisnya, contohnya sebelah kiri tempat bakwan, tengah tempat tempe, dan sebelah kanan tempat tahu\\
Contoh penerapan database pada zaman sekarang yaitu dalam sebuah perusahan membuat tabel berdasarkan proyek yang akan dibuat.contohnya pembuatan aplikasi toko.di dalam aplikasi toko ter dapat beberapa tabel seperti : tabel  barang, tabel pengguna, tabel pelanggan dan tabel penjualan. dan datanya harus valid.


\end{document}
