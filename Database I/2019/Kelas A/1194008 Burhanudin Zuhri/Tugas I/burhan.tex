\documentclass[a4paper,12 pt]{article}
\usepackage{color}
\usepackage[bahasa]{babel}
\usepackage{graphicx}
\title{\textbf{Tugas Rangkuman Basis Data}\linebreak}
\date{}
\begin{document}
	\maketitle
	\begin{center}
		\includegraphics[width=5cm,height=5cm]{logo.png}
	\end{center}
	\begin{center}
		\textbf{Disusun Oleh :} \linebreak
	\end{center}
	\vspace{0.5 cm}
	\begin{center}
		\begin{center}
			Nama   :   Burhanudin Zuhri\\
			NPM   :  1194008\\
			Kelas   :  D4TI 1A\\ 
		\end{center}

		Untuk Memenuhi Tugas Basis Data I Semester 2 \\
		Dosen Pengampu : Syafrial Fachri Pane, ST., MTI., EBDP. 				\linebreak

		Program Studi D4 Teknik Informatika \\
		Politeknik Pos Indonesia\\
		Bandung\\
		2019/2020\linebreak
	\end{center}
	\newpage
	\begin{flushleft}
		\title{\textbf{Basis Data}\linebreak}

	\par Basis data (database) merupakan kumpulan beberapa informasi 		berupa angka, huruf, gambar, maupun simbol yang tersimpan pada 			komputer secara sistematis. Informasi pada database berbentuk 			tabel dan saling berelasi satu sama lainnya serta dapat 				ditampilkan menggunakan program komputer tertentu. Dalam 				pembuatan aplikasi, database berperan penting sebagai penyedia 			informasi yang telah diolah dan dianalisis sehingga menghasilkan 		aplikasi yang terstruktur dengan baik.\\
	\par Cara kerja databse yaitu dengan mengelompokkan dan 				menyederhanakan semua data yang terdapat pada komputer. 				Pengelompokkan dan penyederhanaan data berfungsi untuk 					menghindari adanya data ganda (redudansi) ketika dinormalisasi 			serta berfungsi untuk merancang tempat penyimpanan yang lebih 			mudah dan efisien ketika diakses oleh pengguna. Aplikasi yang 			digunakan sebagai tempat penyimpanan data atau dan mengaksesnya 		dari DBMS (Database Management System) antara lain MySQL, Oracle, 	dan Microsoft Acces.\\
	\par Konsep database sudah ada sejak zaman dahulu yang merupakan 		konsep himpunan pada matematika, namun penerapan konsep database 		saat itu masih sangat sederhanan (manual) yaitu berupa 					penyimpanan benda-benda penting dalam suatu tempat tertentu 			supaya pemiliknya mudah untuk mengambil benda tersebut, contohnya 	adalah dompet untuk menyimpan uang dan kartu-kartu penting serta 		tas untuk menyimpan buku dan peralatan tulis sekolah. Berbeda 			dengan zaman sekarang yang sudah menggunakan teknologi digital 			dan memiliki sistem yang terstruktur sehingga tidak memerlukan 			tempat fisik yang berukuran besar dan mudah untuk diakses 				penggunannya karena dibuat lebih sederhana, contohnya adalah 			pencatatan penduduk dan pencatatan pegawai.\\

\end{flushleft}
\end{document}