\documentclass[12pt]{book}
\title{Rangkuman}
\author{Akmal Restu}
\date{27/02/2020}

\begin{document}
\begin{flushleft}
NPM		: 1194003\\
Nama  	: Akmal Restu Dewantoro\\
Kelas 	: D4 Teknik Informatika 1A\\
Tugas	: Rangkuman Materi Database\\
Dosen	: Syafrial Fachri Pane, S.T., M.T.I.,EBDP
\end{flushleft}
\hrule
\begin{flushleft}
	
1. Database ( Basis Data )\\
   Basis data adalah kumpulan data yang disimpan secara sistematis dan terstruktur dalam satu tempat dan dapat di manipulasi / diolah sesuai dengan kebutuhan. fungsi basis data adalah untuk mengorganisasi dan menghindari duplikasi data (redudansi).
   \newline \linebreak
2. Normalisasi Database\\
   Normalisasi Database merupakan teknik untuk mengolah data pada database agar lebih terstruktur dan memastikan agar tidak ada duplikasi data.
   \newline \linebreak
3. DBMS\\
   DBMS ( \textit{Database Management System} ) Merupakan sistem \textit{software} / aplikasi untuk mengkontrol , mengolah dan mengakses data secara praktis dan efisien. berikut ini adalah beberapa daftar \textit{software} DBMS : \\
   	A. Mysql\\
	B. Maria DB\\
	C. SQL SERVER\\
	D. Oracle dan lain sebagainya.
   \newline \linebreak
4. Relasi\\
   Relasi merupakan hubungan antar tabel untuk mendapatkan data yang lebih rinci.
   \newline \linebreak
5. SQL\\
   SQL ( \textit{Structured Query Language} ) adalah bahasa yang digunakan untuk memanipulasi data pada database , baik itu mengakses , membuat , mengedit ataupun menghapus. sederhananya SQL adalah sebuah \textit{script} / kode yang gunanya untuk mengintruksikan pengolahan data pada database.
   
\end{flushleft}


\end{document}