\documentclass{article}
\usepackage[utf8]{inputenc}

\usepackage{natbib}
\usepackage{graphicx}

\title{TUGAS RANGKUMAN BASIS DATA I}
\author{Nama : Artha Glory Romey Manurung \\ \\ Npm : 1194005 \\ \\ Prodi/Kelas : D4 TI 1A \\ \\ Dosen Pengampu : Syafrial Fachri Pane, S.T., M.T.I.,EBDP}
\date{Kamis, 27 Februari 2020}

\begin{document}
\maketitle

\section{Pengertian Basis Data (Data Base)}
\paragraph{}
Basis data atau Data Base adalah kumpulan beberapa informasi berupa data, angka, huruf, atau gambar yang di susun secara terstruktur atau sistematis agar tidak terjadi pengulangan data (redudansi).
\paragraph{}
Basis data pertama kali di temukan oleh Charles Brachman. RDBMS (Relational Data Base Management System) adalah program yang sistem kerjanya terdiri dari banyak tabel yang mempunyai relasi dari satu tabel ke tabel yang lainnya.

\section{Fungsi Basis Data}
\begin{itemize}
    \item Untuk memudahkan kita saat mengakses informasi.
    \item Penyimpanan data lebih efektif dan efisien karena tidak memerlukan banyak ruang dan bisa di back up.
    \item tidak terjadinya data ganda/redudansi data.
    \item Pengamanan data terhadap penambahan, pengerusakan, pencurian atau gangguan yang lainnya.
\end{itemize}

\newpage
\section{Jenis-jenis Basis Data}
\begin{itemize}
    \item MySQL
    \item Oracle
    \item Firebird
    \item MongoDB
    \item MariaDB
    \item Microsoft (Office) Access
\end{itemize}

\section{ Penerapan Basis Data }
\paragraph{}
Sebelum ditemukannya teknologi komputer, manusia sudah menggunakan basis data tetapi secara MANUAL. Contoh penyimpanan data secara manual seperti perpustakaan yaitu dengan cara mengelompokan setiap data dan menyusunnya ditempat/rak yang ada agar data-data tersebut bisa tersusun dengan rapih. Tetapi cara seperti itu tidak efektif, karena memerlukan banyak ruang dan harus mengingat tempat-tempat data tersebut di tempatkan.
\paragraph{}
Pada Zaman modern seperti sekarang ini kita tidak perlu repot-repot melakukan hal semacam itu, karena sistem penyimpanan atau basis data sekarang menggunakan SISTEM. Yang berarti kita bisa membuat penyimpanan data secara digital.
\paragraph{}
Hal ini sangat menguntungkan kita sebagai pengguna karena saat kita ingin mengakses data-data tersebut kita hanya tinggal mengaksesnya dari komputer dan lebih memudahkan kita ketika ingin membawa data-data tersebut seperti dengan cara menyimpannya di memori penyimpanan seperti dalam flashdisk, hardisk, memory card, dan lain-lainnya.

\end{document}