\documentclass[a4paper,12pt]{article}
\usepackage[utf8]{inputenc}
\usepackage{color}
\usepackage{graphicx}
\usepackage[bahasa]{babel}
\newcommand{\hilight}[1]{\colorbox{green} {#1}}

\title{\large{\textbf{Tugas Pertama DataBase}\\}Rangkuman Mengenai Pertemuan Pertama Tanggal 27 February 2020}
\date{}
\begin{document}
\maketitle
\begin{center}
    \includegraphics[width=5cm]{logo.png}
\end{center}
\vspace{-0.1cm}
\begin{center}
\begin{tabular}{ c c l }

  Nama & : & Eni Lestari \\
  Npm & : & 1194012 \\
  Kelas & : & 1-A \\
  Prodi & : & D4 Teknik Informatika \\
  Dosen Pengampu & : & Syafrial Fachri Pane,S.T.,M.T.I.,EBDP

\end{tabular}
\vfill
\textbf{PROGRAM DIPLOMA IV TEKNIK INFORMATIKA}\\
\textbf{POLITEKNIK POS INDONESIA}\\
\textbf{2020}\\
\end{center} 

\newpage 

	\textbf{I. Apa Itu DataBase}
	\vspace{-0.3cm}
	\paragraph{}
			\hilight{Basis Data} merupakan kumpulan data yang dapat berupa \\ \hilight{Angka, Huruf, Gambar, Symbol} yang mana data tersebut dikumpulkan di satu tempat dan data nya harus sesuai atau nyata, dan data tersebut nanti nya harus saling berelasi.
			
	\vspace{0.8cm}
	\textbf{II. Apa Tujuan DataBase}
	\begin{enumerate}
			\item Dengan \hilight{Mengelompokkan data} untuk mempermudah identifikasi atau mencari data, database menyiapkan data yang sesuai dengan permintaan user terhadap suatu informasi dengan cepat dan akurat. 
			\item Menghindari data ganda atau \hilight {Redudansi}
			\item Memudahkan akses, penyimpanan data,mengedit dan menghapus data 
	\end{enumerate}
	\vspace{0.8cm}

	\textbf{III. Macam-macam Database}
	\begin{itemize}
		\item MySQL 
		\item dBase
		\item Oracle
		\item Visual Foxpro
	\end{itemize}

	\vspace{0.8cm}
	
	\textbf{IV. 2 Tipe Data Base }\\
	\vspace{-0.3cm}
	\begin{itemize}
	\item \textbf{Data Base Manual (jaman dahulu) }
	\paragraph{}
		Pada jaman dahulu, \hilight{database} telah diterapkan akan tetapi masih meggunakan cara \hilight{manual}, karena pada saat itu belom berkembangkan teknologi, antaralain contohnya \hilight{laptop}. Namun dari cara penerapan database secara manual ini, sangat banyak kekurangannya dan \hilight{tidak fleksibel}, contoh kecilnya itu misalkan ada seseorang yang ingin memperbaiki atau memperpanjang KTP, akan tetapi para petugas harus singgap mencari data yang sesuai dengan orang tersebut yang mana pastinya data yang berbentuk lembaran kertas itu telah bertumpuk di gudang atau di map (sudah tidak fleksibel lagi untuk dicari), sehingga dari contoh tersebut juga telah terbukti mulai munculnya kekurangan, dan \hilight{memakan waktu yang lama} untuk pencarian data nya.
	  
	\paragraph{}
		Contoh yang kedua yaitu apabila ada orang yang berniat jahat akan berkas-berkas tersebut, atau juga datangnya musibah, contonya kebakaran, maka para petugas yang pastinya \hilight{akan kesulitan dalam mengangkut atau
		menyelamatkan berkas-berkas} tersebut. Dan masih banyak lagi contoh-contoh kekurangan lainnya untuk data base manual ini.
	
	\item\textbf{ Data Base System (Sekarang) }
	
	\paragraph{}
		Selain berkembangnya teknologi, maka semakin berkembang juga tipe database, antara lain data \hilight{base system}, Pada data base system ini, sangat banyak perubahan dan peningkatan dari data base manual, 
		antaralain yang tertera atau yang saya cantumkan pada tujuan diatas.
	\end{itemize}


\end{document}