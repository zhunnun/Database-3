\documentclass[a4paper,12pt]{article}
\usepackage{graphicx}
\title{RANGKUMAN}
\author{Helmi Salsabila}
\begin{document}
	\begin{center}
			\textbf{TUGAS 1}\\
		\vspace{0.4cm}
			\textbf {Rangkuman Basis Data}\\
		\vspace{1cm}
			\includegraphics [width=5cm] {logo.png}
		
		\vspace{1cm}
			\textbf{Dosen Pengampu}\\
				Syafrial Fachri Pane, S.T.,M.T.I.,EBDP\\

		\vspace{0.4cm}
			\textbf{Disusun Oleh}\\
				Helmi Salsabila\\(1194018)
		\vspace{3cm}
		\\PROGRAM DIPLOMA IV TEKNIK INFORMATIKA\\POLITEKNIK POS INDONESIA\\BANDUNG\\2020 

\newpage
\end{center}

		\textbf{I. Pengertian}
	\vspace{-0.3cm}
		\paragraph{}
				Basis Data(Data Base) merupakan kumpulan informasi yang di dalam informasi tersebut bisa berupa 		 gambar,angka, huruf, symbol dll yang sudah tersusun berdasarkan analisis, nyata/fakta serta dapat dibuktikan.

	\vspace{0.8cm}
		\textbf{II. Tujuan Data Base}
		\paragraph{}
\vspace{-0.3cm}
			1.	Merekap dan menyiapkan Data\\
\vspace{-1cm}
		\paragraph{}
			2.	Menghindari data ganda(redudan)\\
\vspace{-1cm}
		\paragraph{}
			3.	Menjaga data agar tidak hilang dsb.\\
\vspace{-1cm}
		\paragraph{}
			4.	Memudahkan untuk mengakses data  seperti mengupdate data, 
		\paragraph{}
\vspace{-0.6cm}
				penyimpanan dan menghapus/mentiadakan data\\
		\paragraph{}
\vspace{-1cm}
			5.	Untuk menghilangkan data penyimpanan yang masih berbentuk
		\paragraph{}
\vspace{-0.6cm}
				manual/masih menggunakan kertas dan berkas-berkas lainnya 

	\vspace{0.8cm}
		\textbf{III. Jenis Data Base}
		\paragraph{}
			1.	SQL(Lebih ke Strukturnya)
	\paragraph{}
\vspace{-0.3cm}
		Dapat memanipulasi dan mengupdate data serta menjalankan query untuk memperbaharui/mengapdate dan menghapus 			data.\\
		Contoh: MySQL, Oracle, dll.
	\paragraph{}
			2.	NoSQL(Lebih ke Kecepatanya)
	\paragraph{}
\vspace{-0.3cm}
		Memiliki data skema fleksibel untuk membuat aplikasi modern serta menggunkaan model data.\\
		Contoh: MonggoDB, CouchDB dll. 

	
\end{document}