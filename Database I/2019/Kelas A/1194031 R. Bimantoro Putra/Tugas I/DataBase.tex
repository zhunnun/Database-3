\documentclass[10pt]{article}
\title{DATA BASE}
\author{}

\usepackage [bahasa]{babel}
\begin{document}
\maketitle
\begin{center}
\begin{tabular}{ll}
Nama 				&: \textbf {R. Bimantoro Putra}\tabularnewline\\
NPM       			&: \textbf {1194031}\tabularnewline\\
Kelas 				&: \textbf {D4 TI 1A}\tabularnewline\\
Materi 				&: \textbf {Data Base}\tabularnewline\\
Dosen Pengampu		&: \textbf {Syafrial Fachri Pane, ST., MTI., EBDP.}\tabularnewline\\
\end{tabular}
\newline
\newline
\newline
\newline
\newline
\newline
\newline
\textbf {POLITEKNIK POS INDONESIA} \linebreak
TEKNIK INFORMATIKA\linebreak
2019/2020
\end{center}
\newpage
\begin{abstract}
\section{Database}
\subparagraph{} Data Base sebenarnya sudah ada pada zaman dulu sebelum adanya teknologi. Hanya saja, Data Base manual. contohnya seperti Lemari untuk mengelompokkan Pakean, Tas untuk mengelompokkan barang barang, dll.
\subparagraph{}Data Base merupakan kumpulan data yang disimpan secara sistematis didalam komputer yang dapat diolah atau dimanipulasi menggunakan perangkat lunak ( Program aplikasi ) untuk menghasilkan informasi . Kumpulan datanya berupa angka, huruf, bilangan, dll.
\subparagraph{} Semua informasi harus memiliki sumber yang jelas dan nyata.Cara agar data tidak Redudanci dengan cara di Normalisasi. Data dikumpulkan , Diolah untuk disampaikan lagi  menjadi data yang mudah di mengerti dan jelas. Data Base berasal dari Himpunan ( Matematika ) yang memiliki hubungan atau saling berelasi antara satu sama lain. 
\subparagraph{}Tujuan dikelompokkannya data agar tidak terjadinya Redudanci ( Data ganda ). Untuk mengumpulkan data dengan sistematis dengan cara :
  -. Analisis
  -. Penelitian
  -. Pembuktian 
  -. Dll.
  \subparagraph{}Adapun beberapa perangkat lunak yang digunakan antara lain :
  -. MySql
  -. MongoDB
  -. Oracle
  -. Firebase
  -. Dll.
  \section{Kesimpulan}
  \subparagraph{}Data Base merupakan kumpulan data yang disimpan secara sistematis yang diolah dengan Perangkat lunak . Perangkat lunak yang dapat digunakan untuk membuat database adalah MySql, MongoDB, Oracle,Firebase, dll.
  
  
 
\end{abstract}
\end{document}