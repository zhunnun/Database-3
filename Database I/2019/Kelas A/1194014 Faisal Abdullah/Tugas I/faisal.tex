\documentclass[10pt]{article}
\usepackage[latin1]{inputenc}

\title{Rangkuman Materi Basis Data I}
\author{Faisal Abdullah }
\date{27 Februari 2020  }

\usepackage{natbib}
\usepackage{graphicx}

\begin{document}

\maketitle
\begin{center}
    \includegraphics[]{download.png}
\end{center}

\newpage

\section*{Basis Data}
\paragraph{}Basis data adalah sekumpulan data yang bisa berupa angka,huruf,gambar yang saling berhubungan dan faktual kemudian disimpan dalam suatu tempat agar terstruktur 
\paragraph{}Basis data dibagi menjadi dua yaitu manual dan digital.

\section*{Kelebihan Database digital}
\begin{itemize}
  \item Menghindari data agar tidak redudansi/ data ganda
  \item Menghindari data agar tidak memakan lebih banyak ruang
  \item Menghindari data agar tidak rusak karena kesalahan manusia
  \item Agar mudah dicari
  \item Memudahkan data agar mudah di back up
 \end{itemize}
 
  \section*{Kekurangan Database digital}
\begin{itemize}
  \item Data bisa diretas oleh seseorang
 \end{itemize}

 
 \section*{Kekurangan Database manual}
\begin{itemize}
  \item Data lebih memakan banyak ruang
  \item Data bisa lebih mudah rusak karena kesalahan manusia
  \item Bisa terjadi redudansi
  \item Sulit dicari
  \item Data sulit di Back up
 \end{itemize}
 
 \section*{Software Untuk Basis Data}
 \begin{itemize}
  \item MySQL
  \item Oracle
  \item Microsoft Acces
  \item Firebird
 \end{itemize}
 
\newpage
\section*{Tahapan Normalisasi}
\paragraph{Tahapan normalisasi dilakukan agar data tidak redudansi/data nya ganda}

\section*{Table}
\paragraph{Table adalah kumpulan data yang ada di dalam baris dan kolom}

\section*{Hubungan cloud dan Database}
\paragraph{Database disimpan di cloud kemudian apabila data ingin diambil,pengguna tinggal download}

























\end{document}
