\documentclass[a4paper,12 pt]{article}
\usepackage{color}
\usepackage[bahasa]{babel}
\usepackage{graphicx}
\title{\textbf{Rangkuman}\linebreak \\ \textbf{Basis Data}\linebreak}
\date{}
\begin{document}
\maketitle
\begin{center}
\includegraphics[width=5cm,height=5cm]{poltek.png}
\end{center}
\begin{center}
\textbf {Basis Data} \linebreak
\end{center}
\vspace{0.5 cm}
\begin{center}
\begin{tabular}{11}
Nama & : Alvaro Daniel Bamba \\
NPM & : 1194028\\
Kelas & : D4TI 1B\\
\end{tabular}
\newline
\newline
\newline
Untuk Memenuhi Tugas Basis Data \\
Dosen Pengampu: Syafrial Fachri Pane, ST., MTI., EBDP. \linebreak
\newline

Program Study of Informatics Engineering \\
\textit {Politeknik Pos Indonesia}
\linebreak
Bandung 2020 \linebreak
\end{center}
\newpage


Basis data adalah kumpulan data yang saling berelasi dan dibuat bedasarkan fakta. Konsep basis data dibuat berdasarkan kejadian di kehidupan sehari hari, seperti contohnya lemari tempat kita menyimpan barang barang, yang mana hal tersebut memilik kendala. Beberapa diantaranya adalah penyimpanan yang terbatas, biaya yang besar, dan memakan ruangan yang besar juga, ditambah lagi untuk melakukan pencarian dibutuhkan waktu yang lama dan ada kemungkinan kehilangan, sehingga data yang disimpan tidak aman.
\newline
\newline
Hal tersebut mendorong diciptakannya basis data. Basis data mengalami normalisasi yang mana bertujuan untuk mencegah terjadinya redudansi atau pengulangan data. Oleh karena itu software ms excel meskipun menggunakan konsep data base yang dapat mengumpulkan data, ms excel tidak termasuk kedalam database karena dalam ms excel dapat terjadi pengulangan.  
\newline
\newline
Fungsi database adalah untuk mengelompokkan data sesuai tipenya, yang mana hal tersebut dapat memepermudah dalam identifikasi data, basis data juga dapat mengatasi beberapa masalah penyimpanan pada kehidupan sehari hari seperti diperlukannya biaya dan ruangan yang besar.selain itu basis data juga di perlukan dalam pembuatan aplikasi atau web sebagai penyimpanan data, yang  mana basis data adalah komponen penting dalam sebuah aplikasi atau web yang jika database tersebut di ganti sedikit saja, dapat mengakibatkan error pada aplikasi atau web yang di buat.

\end{document}

