\documentclass[12pt , a4paper]{article}
\title{Rangkuman Mengenai Materi Basis Data}
\author{}

\date{}
\usepackage{graphicx}

\begin{document}
\begin{titlepage}
\maketitle
\thispagestyle{empty}

\vspace{0.5cm}
\begin{center}
\includegraphics[width=8.5cm, height=8cm]{poltekpos.png}
\end{center}
\vspace{0.5cm}
\begin{center}
Diajukan untuk memenuhi tugas mata kuliah Basis Data\\
\vspace{12px}
Dosen Pengampu:\\
Syafrial Fachri Pane, ST., MTI., EBDP.
\vspace{12px}

Oleh:\\
Esadhira Giovany Syuhada\\
1194044
\vspace{14px}

\textbf{PROGRAM DIPLOMA IV TEKNIK INFORMATIKA}\\
\textbf{POLITEKNIK POS INDONESIA\\}\textbf{BANDUNG}\\
\textbf{2020}
\end{center}
\end{titlepage}


\newpage
\maketitle
\section{Basis Data}
\paragraph{}
	\indent Basis Data merupakan kumpulan suatu data atau informasi yang di simpan dalam suatu tempat yang dipilah sesuai dengan tipe data yang saling berhubungan. Data dapat disimpan dalam suatu base sesuai dengan kapasitas storage yang tersedia.\\
	\indent Contoh simple nya yang diterapkan dalam kehidupan sehari hari yaitu Lemari. Lemari digunakan tempat penyimpanan untuk pakaian sesuai dengan kapasitasnya. Kita tidak dapat menyimpan barang yang bukan sesuai dengan wadahnya. Contohnya kita ingin menyimpan sebuah mobil didalam lemari. Jika kapasitas lemari sudah terisi penuh maka tidak akan bisa disimpan lebih banyak pakaian lagi. Untuk itu,kita harus membeli 1 buah lemari lagi yang baru tetapi itu tidaklah efisien karena alangkah bagusnya jika kita mengganti lemari tersebut dengan kapasitas yang dapat memyimpan lebih banyak pakaian lagi agar kita tidak kesulitan dalam mencari satu buah pakaian.
	\\ 
	\indent Fungsi dari database : 

\begin{itemize}
\item Memudahkan dalam menyimpan data
\item Menghindarinya redudansi data(pengulangan data) Menggunakan teknik normalisasi
\item Menghemat budget
\item Efisiensi
\item Memudahkan untuk berbagi data
\item Menjaga Keamanan Data
\item Memudahkan untuk mencari suatu data
\item Menghemat Penyimpanan
\item Dapat membackup data jika terjadi suatu kecelakaan
\end{itemize}

\indent Komponen Komponen yang harus diperhatikan dalam membuat database: 
\begin{enumerate}
\item Hardware.\\
\indent Spesifikasi Komputer untuk mengolah data dibutuhkan CPU,Memory,disk,storage yang memadai untuk menunjang pengelolaan database itu sendiri.
\item DBMS  (Database Management System).\\
Dibutuhkan juga software untuk mengelola database seperti Mysql,MS-Access,Oracle dll. 
\end{enumerate}
\end{document}