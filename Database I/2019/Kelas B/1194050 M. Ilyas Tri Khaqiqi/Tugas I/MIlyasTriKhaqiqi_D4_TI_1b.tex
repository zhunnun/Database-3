\documentclass[a4paper,12 pt]{article}
\usepackage{color}
\usepackage{graphicx}
\usepackage[bahasa]{babel}

\title{{Rangkuman}}
\author{Basis Data}
\date{24/02/20}

\begin{document}
\maketitle

\begin{center}
\includegraphics[width=6cm]{poltekpos}
\end{center}

\begin{center}
\author{Disusun Oleh :}
\end{center}
\begin{center}
\author{Nama : M. Ilyas Tri Khaqiqi}
\end{center}
\begin{center}
\author{Npm : 1194050}
\end{center}
\begin{center}
\author{Kelas : D4 TI 1B}
\end{center}
\vspace{1cm}
\begin{center}
\textbf{Program Diploma IV Teknik Informatika}
\end{center}
\begin{center}
\textbf{Politeknik Pos Indonesia}
\end{center}
\begin{center}
\textbf{Bandung}
\end{center}
\begin{center}
\textbf{2020}
\end{center}




\newpage
\section{Basis Data}
\subsection{Pengertian}
	basis data adalah tempat penyimpanan sekumpulan nilai yang di susun berdasarkan typenya masing-masing yang nantinya harus di normalisasi agar menghindari redudansi agar efektif dalam penggunaannya. 
\subsection{Latar belakang di buatnya basis data}
	Basis data ini suatu alternatif yang paling efektif dalam penyimpanan data-data yang penting karena lebih aman dalam menghadapi resiko yang banyak merugikan seperti: faktor bencana, penyimpanan data yang membutuhkan ruangan yang besar, dan kesulitan dalam pencarikan data dsb. dan faktor permasalahan di ataslah bukti dari basis data tercipta yang awal mulanya dari permasalahan yang telah di sebutkan tadi.
\subsection{Implementasi basis data dalam kehidupan}
	implementasi ini hanya sekedar penerapan dari beberapa konsep basis data yang hanya menyerupai konsepnya namun bukan basis data yang sesungguhnya. Ini adalah suatu contoh yang terdapat pada barang yang berada di sekitar kita karena mengimplementasikan konsep pengelompokan, terstruktur dll seperti : kulkas, lemari, gudang, dll
\subsection{Pentingnya memahami basis data}
bagi seorang programmer konsep-konsep basis data adalah suatu hal yang penting karena dalam struktur pemogramman ANSI yang pertama bukan programmer melainkan analysis, database, lagu programmer. dan apabila suatu website atau program yang lainya tidak memakai suatu basis data atau database program tersebut akan sulit untuk dilakukan suatu pembaharuannya. 
\newpage
\subsection{Hal harus di perhatikan dalam pembuatan basis data}
pertama kali anda harus lakukan dalam pembuatan suatu basis data adalah menganalisis suatu hal yang akan anda buat suatu basis datanya,lalu dalam pembuatannya anda juga harus memahami suatu hubungan-hubungan yang akan tercipta saling berelasi di dalam suatu basis datanya, yang ketiga anda harus mempunyai suatu DBMS dalam pembuatan basis datanya. Seperti : MySql, Oracle, Firebird dsb.
\subsection{Kesimpulan}
Basis data atau kerap kali di sebut data base adalah suatu tempat penyimpanan sekumpulan data berdasarkan typenya yang telah melalui normalisasi agar terhindar dari redudansi menjadi solusi dari berbagai persoalan di massa lampau yang telah terjawab solusinya dengan basis data yang berfungsi sangat efektif.
\end{document}
