\documentclass[12pt]{article}
\usepackage{amsmath}
\usepackage{graphicx}
\usepackage{hyperref}
\usepackage[latin1]{inputenc}
\title{Rangkuman Basis Data Pertemuan 1}
\date{}
\begin{document}
\begin{titlepage}
\maketitle
\thispagestyle{empty}

\vspace{0.5cm}
\begin{center}
\includegraphics[width=5cm, height=5cm]{poltekpos.png}
\end{center}
\vspace{0.5cm}
\begin{center}
Diajukan untuk memenuhi tugas mata kuliah Basis Data\\
\vspace{12px}
Dosen Pengampu:\\
Syafrial Fachri Pane, ST., MTI., EBDP.
\vspace{12px}

Oleh:\\
Muhammad Rafly Fachrian Al Bantani\\
1194058
\vspace{14px}

\textbf{PROGRAM DIPLOMA IV TEKNIK INFORMATIKA}\\
\textbf{POLITEKNIK POS INDONESIA\\}\textbf{BANDUNG}\\
\textbf{2020}
\end{center}
\end{titlepage}


\newpage
\maketitle

Basis Data adalah tampat penyimpanan kumpulan data sesuai dengan kategorinya atau klasifikasi yang sama antar data.

Basis data di pelajari karena efisiensinya untuk menyelesaikan suatu masalah dari kebiasaan dan permasalahan yang sering ditemui pada kehidupan sehari hari. 

Ada pun cara kerjanya dengan cara normalisasi untuk menghindari redundansi perulangan data atau penumpukan data dan juga harus terstruktur agar saat mencari data lebih mudah dengan mencari sesuai klasifikasinya saja.

Software yang biasa digunakan untuk mengolah dan menerapkan basis data disebut sistem manajemen database (Database Management System, DBMS)contohnya yaitu mysql,oracle

Basis data mengalami masa peralihan dari penerapan basis tradisional ke basis digital.

Contoh penerapan basis data tradisional

\begin{itemize}
  \item Lemari : karena lemari merupakan media penyimpanannya dan pakaian berupa datanya yang mempunyai nilai yang berbeda dan klasifikasi yang berbeda juga bisa dilihat dengan dibedakannya tempat baju dengan celana karena klasifikasinya pun berbeda.
  \item Perpustakaan : perpustakaan juga merupakan media untuk penyimpanan buku - buku yang dimana buku buku ditata dan diatur sesuai dengan pengklasifikaiannya.
  \item Dompet : Merupakan wadah penyimpanannya dan uang dan kartu dan sebagai halnya merupakan suatu data yang ditampungnya.
        \hspace*{3cm}
            
Contoh penerapan basis digital

  \item Marketplace : tempat berkumpulnya pedagang yang memudahkan penjualnya untuk mencari kebutuhan yang diinginkan sesuai kebutuhan tanpa harus menghabiskan waktu banyak.
  \item Data Server : untuk menampung berbagai macam data misal dipemerintahan dengan adanya itu dapat mempermudah dan efisien untuk mencari nama nasabahnya.
  \item Hardisk : karna perkembangan sangat pesat hardisk yang kecil pun bisa menyimpan foto foto atau file kenangan dari kamu kecil sampai besar dengan penyimpanan yang sangat besar.
\end{itemize}

Pada intinya Basis data dapat mempermudah dan lebih efisien untuk melakukan kehidupan sehari hari.

  \begin{itemize}
  \item Efisien penyimpanan karena database bisa banyak mencakup data yang tidak bisa kita bayangkan seberapa banyak datanya.
  \item Efisien waktu karena untuk mencari sebuah data kita tidak perlu susah payah mencari tinggal kita sortir dan langsung mendapatkan data yang kita inginkan.
  \item Tidak seperti penyimpanan tradisional yang mudah rusak database tidak mudah rusak dan sekalipun rusak systemnya langsung melakukan auto backup data tersebut.
  \item Bersifat sangat aman karena memungkinkan hanya kita yang dapat mengaksesnya.


Ada 3 hal yang wajib kita perhatikan untuk membuat sebuah website yaitu ada 


ANALISIS lalu DATABASE dan yang terakhir PEMROGRAMAN

Setiap bagian mempunyai peran khusus masing masing yang tak bisa dihilangkan bila salah satu dihilangkan mungkin kurang lengkapnya suatu website tersebut dan Database merupakan bagian yang bisa dibilang sangat penting karena bila databasenya saja sudah tidak baik maka selanjutnya pun mungkin akan berantakan.


\end{itemize}
\end{document}
