\documentclass[a4paper, 12pt]{article}

\usepackage{babel}
\usepackage{enumitem}
\usepackage{times}
\usepackage{graphicx}
\usepackage{geometry}
	\geometry{left = 4cm, top = 4cm, right = 3cm, bottom = 3cm}
\usepackage{float}
\usepackage{setspace}
	\setstretch{1.5}
\usepackage{listings}
\usepackage{hyperref}


\begin{document}
\title{\huge\textbf{Rangkuman Database}}
\date{}

\maketitle




\begin{center}
\vspace{4cm}
Lunetta Ivania Sidora\\
D4 TI 1B\\
1.19.40.48\\
\vspace{4cm}
\textbf{PROGRAM DIPLOMA IV TEKNIK INFORMATIKA} \linebreak
\textbf{POLITEKNIK POS INDONESIA} \linebreak
\textbf{BANDUNG}\linebreak
\textbf{2020}\\

\end{center}
\newpage

\section{Basis Data}
\par Basis Data itu sendiri terbagi menjadi 2 kata yaitu basis yang diartikan sebagai sebuah tempat berkumpul atau sebuah tempat penyimpanan dan data itu sendiri yang artinya representasi fakta atau dunia nyata yang dapat diambil berupa gambar , informasi, huruf, dan angka. 
\par Secara rinci dapat dijabarkan, Basis Data atau Database itu sendiri adalah sebuah kumpulan data berupa angka, huruf, gambar serta fakta yang dikumpulkan dalam satu tempat atau satu wadah dimana datanya tersebut diambil berdasarkan fakta dan yang pastinya data-data tersebut harus saling berelasi.
\par Database berfungsi untuk mengelompokkan sekian banyak data untuk mempermudah identifikasi data serta menjaga kualitas data dan informasi yang sesuai. Fungsi dari database yang lainpun bisa dilihat dari penggunaan efisiensi waktu dan tempat untuk menampung sekian banyak data yang nantinya akan diterima dan untuk mempermudah kita dalam mencari data-data yang lama. 
\par Biasanya Database itu selalu menerapkan fungsi normalisasi. Dimana normalisasi itu sendiri fungsinya untuk menghindari Redudency. Sedangkan yang dimaksud Redudency itu seperti sebuah perulangan atau penghindaran data ganda.
\par Pada Database kita juga bisa memberikan contoh penerapan dalam kehidupan sehari-hari. Seperti sebuah, lemari, dompet, ataupun sebuah tas. Kenapa semua benda-benda itu bisa dibilang sebagai contoh dari Database? Karena sebuah lemari itu bisa dibilang sebagai suatu tempat yang dapat menampung benda-benda yang seharusnya berada di dalamnya. 
\end{document}

