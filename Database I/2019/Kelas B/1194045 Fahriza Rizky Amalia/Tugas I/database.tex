\documentclass{article}
\usepackage[utf8]{inputenc}

\begin{document}
\title{\huge\textbf{Rangkuman Basis Data}}
\date{}


\maketitle



\begin{center}
\vspace{4cm}
Fahriza Rizky Amalia\\
D4 TI 1B\\
1.19.40.45\\
\vspace{4cm}
\textbf{PROGRAM DIPLOMA IV TEKNIK INFORMATIKA} \linebreak
\textbf{POLITEKNIK POS INDONESIA} \linebreak
\textbf{BANDUNG}\linebreak
\textbf{2020}

\end{center}
\newpage
\section{Database}
\par Basis sebagai tempat penyimpanan berupa :
\par •	Kumpulan
\par •	Gabungan
\par •	Tempat
\par •	Bidang 
\par •	wadah
\newline
\par Data yang harus falid, nyata, fakta dan terstruktur berupa :
\par •	Gambar
\par •	Nilai
\par •	Informasi
\par •	Symbol
\par •	Angka
\par •	Hurufpar
\par •	Tulisan
\par •	kata
\newline
\par Basis data adalah kumpulan informasi yang dikelompokkan atau dikategorikan dalam bentuk table kemudian direlasikan atau dihubungkan antar tabel satu ke tabel lainnya yang saling berhubungan. Dalam basis data harus dilakukan normalisasi agar tidak terjadi redudansi atau pengulangan kata yang tersimpan dalam elektronik tersebut sehingga menjadi data yang terstruktur.
\newline
\par Awal mula terbentuk basis data karena faktor dalam kehidupan sehari-hari dengan tujuan untuk mempermudah kegiatan manusia. Seperti pendataan penduduk dalam suatu wilayah di Jawa Barat agar data-data tersebut lebih aman jika terjadi	kebakaran atau data tersebut hilang maka backup data tersebut masih tersimpan. 
\newline
\par Contoh penerapan basis data yang masih bersifat tradisional :
\par •	Tas
\par •	Lemari 
\par •	Dompet
\par •	Kulkas
\newline
\par Contoh penerapan basis data teknologi :
\par •	KTP atau KTM
\par •	Kasir
\newline
\par Software atau DBMS (Database Management System) yang digunakan seperti Msql 
\newpage
\par Fungsi basis data yaitu :
\par •	Mempermudah identifikasi data 
\par •	Menghindari terjadinya  duplikasi
\par •	Mempermudah dalam menyimpan data
\par •	Mudah mengakses dan mendapatkan data
\par •	Space penyimpanan lebih banyak
\par •	Server lebih aman
\par •	Dapat mengurangi terjadinya redudansi
\newline
\par Kekurangan basis data yang bersifat tradisional :
\par •	Sulit dalam mengakses data
\par •	Space penyimpanan yang terbatas
\par •	Keamanan server lebih rawan

 


\end{document}
