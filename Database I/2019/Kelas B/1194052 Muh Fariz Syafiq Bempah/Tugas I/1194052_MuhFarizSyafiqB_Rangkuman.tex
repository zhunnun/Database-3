\documentclass[a4paper,12 pt]{article}
\usepackage{color}
\usepackage[bahasa]{babel}
\usepackage{graphicx}
\title{\textbf{Rangkuman Basis Data}\linebreak}
\date{}
\begin{document}
\maketitle
\begin{center}
\includegraphics[width=5cm,height=5cm]{POLPOS.png}
\end{center}
\begin{center}
\end{center}
\vspace{0.5 cm}
\begin{center}
\begin{tabular}{11}
Nama & : Muh Fariz Syafiq Bempah \\
NPM & : 1194052\\
Kelas & : D4 TI 1B\\
\end{tabular}
\newline
\newline

Untuk Memenuhi Tugas Basis Data \\
Dosen Pengampu: Syafrial Fachri Pane, ST., MTI., EBDP. \linebreak
\newline

Program Studi Teknik Informatika \\
\textit {Politeknik Pos Indonesia}
\linebreak
Bandung 2020 \linebreak
\end{center}
\newpage

Basis data berasal dari 2 kata yaitu “basis” dan “data”. Basis berarti tempat/kumpulan penyimpanan sedangkan data berarti nilai/angka/info. Sehingga dapat disimpulkan, basis data adalah kumpulan data yang saling berhubungan dan disimpan di suatu tempat secara sistematis dan berurut. Dalam prosesnya ada yang disebut dengan redudansi dan normalisasi. Normalisasi adalah suatu proses yang dilakukan untuk menghindari terjadinya redudansi, sedangkan redudansi itu sendiri adalah kejadian dimana data muncul berulang-ulang didalam database yang dapat mengakibatkan pemborosan penyimpanan. Karena itulah normalisasi ini perlu dilakukan dengan tujuan dapat mendeteksi ada atau tidaknya data yang berulang.
\newline
\par Contoh penerapan database dikehidupan sehari-hari yaitu lemari/kulkas. Lemari berfungsi untuk menyimpan banyak pakaian didalam satu tempat yang sama. Tetapi didalam lemari itu terbagi lagi menjadi beberapa bagian untuk mengelompokkan pakaian berdasarkan fungsinya agar lebih terorganisir. Jika semakin banyak pakaian yang ada didalam lemari tersebut, maka kita membutuhkan satu lemari lagi untuk meletakkan pakaian yang mana hal ini memerlukan ruang lebih dan membuat ruangan lebih sempit daripada sebelumnya.
\newline
\par Dari kasus diatas, timbul suatu masalah baru mengenai tempat penyimpanan. Untuk itu kita memerlukan suatu mekanisme untuk dapat menyimpan data dalam jumlah yang banyak tanpa memerlukan ruangan yang besar. Inilah manfaat dari basis data yaitu kita bisa menyimpan banyak data di satu tempat tanpa memerlukan ruangan yang besar untuk menyimpan semua data tersebut.
\end{document}