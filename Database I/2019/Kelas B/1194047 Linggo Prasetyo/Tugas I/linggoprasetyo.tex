\documentclass[a4paper,12 pt]{article}
\usepackage{color}
\usepackage[bahasa]{babel}
\usepackage{graphicx}
\title{\textbf{Rangkuman Basis Data}\linebreak}
\date{}
\begin{document}
\maketitle
\begin{center}
\textbf {Basis Data} \linebreak
\end{center}
\vspace{0.8cm}
\begin{center}
\begin{tabular}{11}
Nama & : Linggo Prasetyo \\
NPM & : 1194047\\
Kelas & : D4TI 1B\\
\end{tabular}
\newline
\newline
\newline
Untuk Memenuhi Tugas Basis Data \\
Dosen Pengampu: Syafrial Fachri Pane, ST., MTI., EBDP. \linebreak
\newline
\newline
\newline
Program Study of Informatics Engineering \\
\textit {Politeknik Pos Indonesia}
\linebreak
Bandung 2020 \linebreak
\end{center}
\begin{flushleft}
\newpage
\begin{document}
\maketitle

\section{Data base}


Basis data atau data base dapat di artikan juga base yang berarti tempat, penympanan, kumpulan atau lokasi dimana data akan di tempatkan, sedangkan data adalah nilai atau informasi yang dapat berupa angka,huruf,symbol dan sebagainya. Konsep data base juga dapat kita jumpai pada aktivitas sehari hari contohnya pada kasus tempat penyimpanan baju, dimana tempat pemyimpanan atau lemari menjadi base yang dapat menyimpan sebuah data yang berupa baju, dan baju dapat di artikan data untuk di simpan. Namun basis data pada sistem mesin akan lebih efisien dan terstruktur sehingga tidak terjadi pengulangan pendataan atau normalisasi data dan agar lebih mudah dalam pencariaan data yang banyak agar mempermudah pekerjaan manusia.


\section{DBMS}
Dbms atau data base management adalah software yang di gunakkan untuk mengakses data base contohnya Oracle, MySQL

\section{Normalisasi}
Normalisasi adalah perintah pada basis data yang di lakukan agar tidak terjadi redudansi, redudansi adalah kejadian dimana data terinput secara ber ulang atau ganda

\end{document}
