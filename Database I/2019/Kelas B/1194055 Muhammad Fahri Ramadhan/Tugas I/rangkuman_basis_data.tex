
\documentclass[a4paper,12pt] {article}
\usepackage {color}
\usepackage{graphicx} 
\usepackage{indentfirst}
\usepackage{enumerate}

\title{Rangkuman Basis Data}
\author{}
\date{}

\begin{document}
\begin{titlepage}
\maketitle
\thispagestyle{empty}

\vspace{0.5cm}
\begin{center}
\includegraphics[width=8.5cm, height=8cm]{poltekpos.png}
\end{center}
\vspace{0.5cm}
\begin{center}
Diajukan untuk memenuhi tugas mata kuliah Basis Data\\
\vspace{12px}
Dosen Pengampu:\\
Syafrial Fachri Pane, ST., MTI., EBDP.
\vspace{12px}

Oleh:\\
Muhammad Fahri Ramadhan\\
1194055
\vspace{14px}

\textbf{PROGRAM DIPLOMA IV TEKNIK INFORMATIKA}\\
\textbf{POLITEKNIK POS INDONESIA\\}\textbf{BANDUNG}\\
\textbf{2020}
\end{center}
\end{titlepage}


\newpage

\maketitle

\section{Pengertian Basis Data}
Basis Data terdiri dari dua kata yaitu basis dan data. Basis dapat diartikan sebagai tempat penyimpanan. Data yang dapat diartikan sebagai objek yang dapat disimpan dalam bentuk huruf, angka, simbol, gambar, teks, suara, atau kombinasinya.
\par Jadi secara istilah basis data adalah kumpulan data yang disimpan dan disortir berdasarkan kategorinya masing-masing. Basis data yang baik adalah basis data yang sudah melalui proses normalisasi.

\section{Tujuan Basis Data}
Basis data diciptakan dengan melihat berbagai permasalahan didalam kehidupan sehari-hari, diantaranya:

\begin{enumerate}[a.]
\item Pencarian sebuah data yang harus memakan waktu jika dikerjakan secara manual. Oleh karena itu perlu ada teknologi yang mampu mencari sebuah data dengan waktu yang efisien.
\item Mengurutkan sebuah data juga jika dilakukan secara manual akan sangat memakan waktu sehingga diciptakanlah teknologi ini.
\item Ketika menginput data secara manual dikhawatirkan terjadi redudansi (pengulangan data) sehingga perlu ada teknologi yang dapat meminimalisir hal tersebut.
\item Tempat penyimpanan yang terbatas. Oleh karena itu dibutuhkan teknolo-gi yang dapat menyimpan data dengan keterbatasan tempat penyimpanan tersebut.
\end{enumerate}

\section{Database Management System}
Database Management System atau yang biasa disingkat DBMS adalah sebuah software yang digunakan untuk mengorganisasikan suatu basis data. Ada beberapa jenis DBMS yang biasa digunakan, diantaranya:
\begin{itemize}
\item MySQL
\item ORACLE
\item Microsoft SQL Server
\item Firebird
\end{itemize}

\end{document}
