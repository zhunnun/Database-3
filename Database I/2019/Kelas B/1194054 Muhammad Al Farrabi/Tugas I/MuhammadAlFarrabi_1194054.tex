\documentclass[a4papper,12pt]{article}
\usepackage{color}
\usepackage[bahasa]{babel}
\usepackage{graphicx}

\title{\textbf{Rangkuman}\linebreak \\ \textbf{Basis Data}}
\date{24 Februari 2020}

\begin{document}
\maketitle

\begin{center}
\includegraphics[width=5cm]{D:/Git/tugas-basis-data/logo.jpg}
\end{center}

\begin{center}
\author(Nama : Muhammad Al Farrabi)
\end{center}
\begin{center}
\author(NPM : 1194054)
\end{center}
\begin{center}
\author(Kelas : D4TI 1B)
\end{center}

\begin{center}
\textbf{Untuk Memenuhi Tugas Basis Data}
\end{center}
\begin{center}
\textbf{Program Studi Tekni Informtika}
\end{center}
\begin{center}
\textbf{Politeknik Pos Indonesia}
\end{center}
\begin{center}
\textbf{Bandung 2020}
\end{center}
\newpage
\section{Basis Data}
\subsection{Pengertian}
	\paragraph{} Basis data adalah tempat untuk menyimpan kumpulan kumpulan data yang disusun berdasarkan type nya masing-masing yang nantinya dinormalisasi agar menghindari redudansi agar efektif dalam penggunaannya.
\subsection{Latar belakang}
	\paragraph{}Basis data adalah suatu alternatif yang efektif untuk menyimpan data-data yang penting karena dapat di back up kembali data yang hilang. Permasalahan penyimpanan data sangat penting karena lebih aman dalam menghadapi banyak risiko buruk seperti: bencana alam ,menghindari risiko kebakaran, kurang nya tempat penyimpanan data tersebut.
\subsection{Impelementasi Basis Data dalam Kehidupan}
	\paragraph{}Implementasi ini hanyalah penerapan beberapa konsep basis data yang hanya menyerupai konsep, tetapi bukan database yang sebenarnya. Ini adalah contoh yang ditemukan dalam hal-hal yang ada di sekitar kita karena mengimplementasikan konsep seperti kehidupan sehari hari seperti: kulkas, lemari, dompet, dan lain lain.
\subsection{Pentingnya Memahami Basis Data}
	\paragraph{}Konsep basis data adalah hal yang penting karena dalam struktur pemrograman ANSI yang pertama bukanlah seorang programmer melainkan analysis, database, lalu programer, dan jika suatu web atau aplikasi tidak memakai suatu basis data atau database maka program tersebut akan sulit melakukan pembaharuan.
\newpage
\subsection{Hal yang harus diperhatikan dalam pembuatan Basis Data}
	\paragraph{}pertama kali anda harus lakukan dalam pembuatan suatu basis data adalah menganalisis suatu hal yang akan anda buat suatu basis datanya, lalu dalam pembuatannya anda juga harus memahami suatu hubungan-hubungan yang akan tercipta saling berelasi di dalam suatu basis datanya, yang ketiga anda harus mempunyai suatu DBMS dalam pembuatan basis datanya. Seperti : MySgl, Oracle, Firebird dsb.
\subsection{Kesimpulan}
	\paragraph{}Basis data atau kerap kali di sebut data base adalah suatu tempat penyimpanan sekumpulan data berdasarkan typenya yang telah melalui normalisasi agar terhindar dari redudansi menjadi solusi dari berbagai persoalan di massa lampau yang telah terjawab solusinya dengan basis data yang berfungsi sangat efektif.
\end{document}
