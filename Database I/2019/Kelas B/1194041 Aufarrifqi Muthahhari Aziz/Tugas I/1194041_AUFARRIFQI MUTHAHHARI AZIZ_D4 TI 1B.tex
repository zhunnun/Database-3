\documentclass{article}
\usepackage[utf8]{inputenc}

\title{RANGKUMAN - BASIS DATA I}
\author{1194041 - AUFARRIFQI MUTHAHHARI AZIZ - D4 TI 1B }
\date{Senin, 24 Februari 2020}

\begin{document}

\maketitle

\section{Basis Data}
Basis Data atau yang biasa disebut dengan Database merupakan suatu kumpulan informasi atau data yang disimpan didalam komputer atau sebuah tempat yang mana kumpulan informasi dan data tersebut disusun secara sistematis agar memudahkan dan mempercepat proses pengambilan suatu informasi didalam database. Database diciptakan dikarenakan kurangnya ruang penyimpanan untuk data, menghemat budget untuk membeli penyimpan data, memudahkan mencari data informasi yang dibutuhkan, dan untuk menyimpan backup-an data. \\

Contoh database dalam kehidupan sehari-hari kita adalah lemari yang berisi berbagai macam pakaian, kulkas yang berisi berbagai makanan dan minuman, dan tas yang berisi berbagai macam barang. Yang disebutkan tadi disebut salah satu contoh database karena pada dalam lemari terdapat berbagai macam jenis pakaian yang kemudian disusun sesuai jenis pakaiannya itu agar disaat ingin dikenakan kita dapat mudah mencari pakaian yang ingin kita kenakan karena sudah tersusun secara sistematis, begitu juga dengan kulkas dan tas. \\

Pada komputer jika kita ingin mengambil dan memasukkan data kita menggunakan software yang bernama Database Management System (DBMS). Contoh aplikasi database adalah MySQL dan ORACLE Didalam database terdapat tabel, jumlah tabel pada tabel lebih baik berjumlah lebih dari satu buah, karena jika semua data dikumpulkan didalam satu tabel maka isi dari tabel tersebut tidak akan sistematis, sehingga kita harus menambahkan tabel pada database dan setiap tabel dibuat saling terhubung.\\

\end{document}