\documentclass[a4paper,12 pt]{article}
\usepackage{color}
\usepackage[bahasa]{babel}
\usepackage{graphicx}
\title{\textbf{Rangkuman}\linebreak \\ \textbf{Basis Data}\linebreak}
\date{}
\begin{document}
\maketitle

\begin{center}
\includegraphics[width=5cm,height=5cm]{logo.jpg}
\end{center}
\begin{center}
\textbf {Basis Data} \linebreak
\end{center}
\thispagestyle{empty}
\vspace{0.5 cm}
\begin{center}
\begin{tabular}{11}
Nama &	: Muhammad Rizal Supriadi \\
NPM & : 1194059\\
Kelas & : D4TI 1B\\
\end{tabular}
\newline
\newline
\newline
Untuk Memenuhi Tugas Basis Data \\
Dosen Pengampu: Syafrial Fachri Pane, ST., MTI., EBDP.  \linebreak
\newline

\newline
Program Studi D4 Teknik Informatika \\
\textit {Politeknik Pos Indonesia}
\linebreak
Bandung 2020 \linebreak
\end{center}

\newpage
\begin{flushleft}
\title{\textbf{1. Pengenalan Basis Data}\linebreak 
\end{flushleft}
\par Basis Data merupakan suatu kumpulan data yang saling mempunyai keterkaitan serta terorganisasi dengan baik sehingga memudahkan untuk melakukan penyimpanan data pada sekala yang cukup besar. Jika datanya sedikit mungkin peran dari basis data ini tidak terlalu berperan penting, karena basis data pada dasarnya diperuntukan untuk menangati data berskala cukup besar.
\par Untuk penerapannya sendiri basis data harus dibuat secara terstruktur dan sebisa mungkin melakukan normalisasi untuk menghindari terjadinya redudansi antara data yang satu dengan data yang lainnya, termasuk juga dalam program aplikasi yang dibuat dengan tujuan untuk bekerja dalam suatu sistem serta untuk memanipulasi data data sehingga lebih mudah memperoleh informasi dari basis data tersebut.
\newline

\begin{flushleft}
\title{\textbf{2. Normalisasi Database}\linebreak 
\end{flushleft}
\par Normalisasi database adalah suatu pendekatan secara sistematis untuk dapat meminimalisir terjadinya redudansi data yang terdapat pada suatu database agar dapat bekerja secara optimal, contohnya ketika kita mempunyai suatu tabel customer dengan field data diri serta terdapat username dan password didalam suatu table tersebut maka untuk melakukan normalisasi kita harus membuat tabel baru dengan nama login untuk memisahkan username dan password sehingga nantinya tabel login ini dipanggil oleh tabel customer tadi.  
\par Normalisasi juga dapat dikatakan sebagai teknik suatu relasi yang dapat digunakan sehingga membentuk struktur yang baik. Yang bertujuan untuk meminimalisir atau menghilangkan redudansi data serta untuk mempermudah mencari data dari kumpulan data yang sangat banyak.
\newline

\begin{flushleft}
\title{\textbf{3. Penjelasan DBMS dan RDBMS}\linebreak 
\end{flushleft}
\par DBMS mempunyai kepanjangan (Database Management System) merupakan suatu kumpulan data dalam jumlah yang sangat banyak yang pastinya sangat membingungkan jika dikelola secara manual oleh karena itu muncul DBMS sebagai konsep management database yang modern untuk mengelola database seperti membuat, mengontrol, memanipulasi serta mengakses database secara efisien. 
\par RDBMS mempunyai kepanjangan (Relational Dabase Management System) merupakan suatu DBMS yang mendukung adanya hubungan atau relationship untuk membuat data menjadi saling terintegrasi, untuk membuatnya maka dua atau lebih tabel harus mempunyai hubungan antara tabel yang satu dengan tabel yang lainnya. Karena pada tabel terdapat primary key lalu dihubungkan dengan kunci tamu atau yang biasa disebuat foreign key.  
\end{document}


