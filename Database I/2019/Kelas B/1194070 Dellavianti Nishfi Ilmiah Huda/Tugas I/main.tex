\documentclass{article}

\usepackage{babel}
\usepackage{enumitem}
\usepackage{times}
\usepackage{graphicx}
\usepackage{geometry}/
	\geometry{left = 4cm, top = 4cm, right = 3cm, bottom = 3cm}
\usepackage{float} 
4

\usepackage{setspace}
	\setstretch{1.5}
\usepackage{listings}
\usepackage{hyperref}

\begin{document}
\title{\huge\textbf{TUGAS DATABASE I\\
Rangkuman DataBase}}
\date{}

\maketitle




\begin{center}
\vspace{4cm}
Dellavianti Nishfi Ilmiah Huda\\
D4 TI 1B\\
1194070\\
\vspace{4cm}
\textbf{PROGRAM DIPLOMA IV TEKNIK INFORMMATIIKA} \linebreak
\textbf{POLITEKNIK POS INDONESIA} \linebreak
\textbf{BANDUNG}\linebreak
\textbf{2020}\\

\end{center}

\begin{document}

\maketitle

\section{DATABASE}\\

\par Dalam pembuatan database harus terstuktur atau teratur dan saling berelasi yang sesuai sehingga database akan lebih mudah dibedakan sesuai tempat penyimpanan masing-masing misalkan dalam sebuah Almari terdapat 2 kotak tempat yang atas dipakai untuk pakaian yang bagus-bagus dan yang bawah disimpan untuk menyimpan pakaian sehari-hari atau dipakai dirumah. Maka dari itu dengan menggunakan teknologi Database akan mempermudah pengelompokkan data berdasarkan fungsi masing-masing.\\

\section{Apa itu Database?}\\
\par Database adalah sebuah informasi atau data-data yang berupa angka,huruf,gambar dan symbol yang dikumpulkan menjadi satu wadah atau tempat dan disimpan secara sistematik, sehingga dapat dengan mudah dicari ataupun diperiksa menggunakan progam komputer untuk mencari informasi dari sebuah data base tersebut.\\

\par Dalam pembuatan Database harus mengetahui Normalisasi yang fungsinya untuk menghindari redudansi atau terjadinya penggulangan pada data tersebut ataupun data ganda dan inkonsistensi data.





\end{document}
