\documentclass{article}
\usepackage[utf8]{inputenc}

\title{Basis Data 1 Rangkuman}
\author{1194063 Rachmatika Intan 1B }
\date{February 2020}

\begin{document}

\maketitle

\section{Pengertian}
Basis Data (tempat penyimpanan)
- Basis Data adalah kumpulan-kumpulan data yang telah di normalisasikan untuk menghindari adanya pengulangan/redudansi data yang sama dan disimpan di dalam komputer secara sistematik sehingga dapat diperiksa menggunakan program komputer untuk mendapatkan data yang diinginkan tersebut.

- Basis : merupakan kumpulan, gabungan, tempat, dan bidang.

- Data : merupakan nilai, gambar, informasi, symbol, fakta, dan waktu.

- DBMS : adalah Data Base Management System
\section{Aplikasi Basis Data}
Aplikasi Basis Data yang biasanya digunakan adalah :

- MySQL (Structured Query Language)

- Oracle
\section{Contoh Basis Data}
Dalam kehidupan sehari-hari :

- Dompet

- Tas

- Kulkas

- Gudang

- Lemari
\section{Tujuan}
Alasan diciptakannya Basis Data sebagai berikut :

- Menghemat Budget

- Mumadhakan untuk mencari data

- Adanya backup data yang disimpan dan akurat

- Efisiensi waktu, ruang penyimpanan, dan kemudahan akses


\end{document}
