\clearpage
\setcounter{page}{1}

\begin{center}
\title{\LARGE \bf Rangkuman Video 1}
\end{center}

\section*{\normalsize 1.Pengertian Oracle Apex} 
\hspace {\parindent}Oracle apex merupakan platform pengembangan kode rendah yang membangun aplikasi pada sebuah perusahaan yang aman dengan fitur kelas dunia yang bisa digunakan dimana saja.
\hspace {\parindent}Dalam apex ada tiga alat utama, yaitu:
\begin{enumerate}
\item Application Builder
Apllication Builder memilliki fungsi untuk membuat dan melihat aplikasi, mengimport aplikasi, mengatur service, mengatur uder aplikasi dan memantau aktifitas yang dilakukan oleh pengguna.
\item SQL Workshop
SQL Workshop memiliki fungsi untuk membuat tabel dan komponen-komponen(dengan menggunakan kode-kode PL-SQL secara manual maupun otomatis),untuk melihat struktur tabel serrta komponenya, mengimpor dan mengeksport script.
\item Utilitas 
Utilitas digunakan untuk melihat report tabel dan juga kompnenya serta hostory aplikasi juga. Kerangka dalam kerja pengembangan aplikasi web database centric ada 3 yaitu:
\begin{itemize}
\item Pengembangan aplikasi web desktop dan seluler
\item memvisualisasikan dan memelihara data pada basis data
\item meningkatkan ketrampilan sql dan kemampuan pada basis data.
\end{itemize}

\section*{\normalsize 2.Jenis Oracle Apex dan Membuat Aplikasi}
Jenis-jenis aplikasi yang cocok untuk apex oracle adalah: 
\begin{enumerate}
\item Aplikasi yang sudah banyak digunakan orang, misalkan aplikasi yang digunakan ribuan banyak orang
\item digunakan utuk mengisi kekosongan sistem.
\item Lebih digunakan untuk mengembangkan proes bisnis yang sudah ketinggalan zaman.
\item Aplikasi pelayanan yang mandiri untuk semua karyawan.
\item Aplikasi yang bisa digunakan oleh perangkat apapun.
\end{enumerate}

\section*{\normalsize 3.Membuat Aplikasi}
\hspace {\parindent}Sebelum membuat aplikasi, untun langkah pertama kalinya yaitu anda harus membuat workspace terlebih dahulu.
\begin{enumerate}
\item Buka link pada apex.oracle.com
\item setelah itu, klik request a free workspace
\item kemudian isi data yang ada di request a workspace, setelah itu tekan next.
\item Setelah masuk pada request workspace lalu tekan.
\item klik next lagi lalu isi data lagi pada request workspace.
\item sis data lagi pada request workspace sampe ada tulisan sukses.
\item setelah sukses, klik app builder.
\item klik create.
\item pilih new application.
\item setelah itu check all nya di klik
\item lalu akan muncul sebuah tampilan klik create
\item klik load data
\item tunggu hingga proses selesai.
\item lalu klik run application.
\item Tunggu sampai kebuka aplikasinya.
\end{enumerate}