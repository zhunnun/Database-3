\chapter{Apex}
\section{Pengertian}
Apex adalah sebuah platform untuk mendevelopment sebuah low-code yang memperbolehkanmu untuk membuat sebuah aplikasi yang aman dan terukur dengan fitur kelas dunia yang bisa di kembangkan dimana saja. Sedangkan low-code sendiri memiliki arti yaitu sebuah framework yang memudahkan kita dalam proses pengembangan suatu aplikasi.
\par 
Applikasi yang cocok menggunakan Oracle apex diantaranya.
\begin{enumerate}
\item Aplikasi skala besar yang diperuntukkan untuk ribuan pengguna
\item Mengisi kekosongan  pada sistem perusahaan
\item Memooderenisasi sistem pendahulunya
\item Aplikasi Self-service untuk semua pekerja
\item Applikasi responsif yang dapat bekerja dalam segala jenis platform
\item Menggantikan penggunaan spreadsheet
\end{enumerate}

Opsi pengembangan.
\begin{enumerate}
\item \textbf{Local} 
\begin{itemize}
\item Install pada pc/laptop anda Oracle Express Edition(XE) atau full version database
\item Upgrade apex ke versi yang dibutuhkan
\item Bisa dikerjakan pada saat offline
\end{itemize}

\item \textbf{On-Premise}
\begin{itemize}
\item Dioperasikan oleh bagian IT perusahaan tersebut
\end{itemize}

\item \textbf{Cloud}
\begin{itemize}
\item Meluncurkan aplikasi internet
\end{itemize}
\end{enumerate}

Kegunaan
\begin{enumerate}
\item \textbf{Apex untuk pelajar} 
\begin{itemize}
\item Memahami pembelajaran mengenai SQL dan Relasi pada database
\item Praktek pengerjaan SQL
\item Mempelajari mendevelop aplikasi
\item Projek akademik
\end{itemize}

\item \textbf{Apex untuk pengajar}
\begin{itemize}
\item Penggunaan SQL dan relational database
\item Pengerjaan menggunakan SQL
\item mendevelop aplikasi
\item mendevelop aplikasi menggunakan framework low code
\item visualisasi data
\end{itemize}

\end{enumerate}


\section{Membuat Aplikasi Dari Spreadsheet}
\begin{enumerate}
\item Log in ke workspace yang sudah dibuat pada laman apex.oracle.com
\item Klik app builder
\item Klik create new app
\item Klik from file, agar data dari spreadsheet yang sudah dibuat dapat diimport kedalam workspace
\item Klik copy and paste
\item Untuk sample set data pilih project and task
\item Klik next
\item Masukkan nama tabel, dalam masalah kali ini menggunakan ( SPREADSHEET )
\item Klik Load data
\item Cek pemberitahuan bahwa 73 baris sudah dimuat, jika sudah klik continue to create application wizard
\item Masukkan nama aplikasi yang akan dibuat
\item Disebelah features klik check all
\item Klik Create application
\item Aplikasi yang sudah dibuat akan ditampilkan di page designer
\item Klik Run application
\end{enumerate}

