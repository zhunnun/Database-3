\chapter{Apex}
\section{Pengertian}
Apex adalah suatu platform untuk mendevelovment sebuah low-code yang digunakan untuk membangun apalikasi yang aman dan terukur dengan fitur kelas dunia yang dapat dikembangkan dimana saja.sedangkan low-code adalah sebuah famework yang memudahkan kita dalan proses penggunaan suatu aplikasi
\par Aplikasi yang mendukung penggunaan apex diantaranya adalah :
\begin{enumerate}
\item Aplikasi skala besar yang diperuntukan untuk ribuan pengguna 
\item Mengisi kekosongan pada sistem perusahaan
\item memoderenisasi sistem pendahulunya
\item Applikasi self-service untuk semua pekerja
\item Applikasi responsif  yang dapat bekerja dalam segala jenis platform
\item mengggantikan penggunaan spreadsheet
\end{enumerate}

Opsi pengembangan :
\begin{enumerate}
\item \textbf {Local}
\begin{itemize}
\item instal pada Pc/anda oracle Express Edition (EX) atau full version database
\item upgrade apex ke versi yang dibutuhkan 
\item Bisa dikerjakan pada saat offline
\end{itemize}

\item \textbf{on-primise}
\begin{itemize}
\item Dioperasikan oleh bagian  IT perusahaan tersebut 
\end{itemize}

\item \textbf{cloud}
\begin{itemize}
\item meluncurkan applikasi internet
\end{itemize}
\end{enumerate}

Kegunaan Apex:
\begin{enumerate}
\item \textbf{Apex untuk pelajar}
\begin{itemize}

\item Memahami pembelajaran menggunakan Sql dan Relasi dapa database
\item Praktek pengerjaan my sql
\item Mempelajari Mendevlop aplikasi 
\item Projek akademik

\end{itemize}


\item \textbf{Apex untuk pengajar}
\begin{itemize}

\item Penggunaan SQL dan relasiona database
\item pengerjaan menggunakan SQL
\item mendevlop aplikasi 
\item mendevlop aplikasi menggunakan framework low code
\item visualisasi data
\end{itemize}

\end{enumerate}

\section{Membuat Aplikasi Spreadsheet}
\begin{enumerate}

\item Log in ke workspace yang sudah dibuat pada laman apex.oracle.com
\item Klik app builder 
\item Klik creat new app 
\item Klik from file, agar data dari spreadsheet yang sudah dibuat dapat diimport kedalam worksapce
\item Klik copy and paste
\item Untuk sample set data pilih project and task
\item Klik next 
\item Masukan nama tabel,dalam masalah kali ini menggunakan ( SPREADSHEET )
\item klik load data
\item Cek pemberitahuan bahwa 73 baris sudah dimuat, jika sudah diklik continue to creat application wizard
\item Masukan nama aplikasi yang akan dibuat 
\item Disebelah features klik chek all
\item klik creat application 
\item Aplikasi yang sudah dibuat akan ditampilkan di page designer
\end{enumerate}