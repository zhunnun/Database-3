
\documentclass[a4paper,12pt]{report}
\title{Tugas Data base}
\author{Rayhan Yuda Lesmana}
\date{23 Oktober 2019}

\begin{document}
\maketitle
\paragraph{Apex}
Application Express(Oracle Apex),suatu alat yang digunakan untuk membangun Aplikasi web-based. Selain itu Oracle Application Express tidak membutuhkan perangkat lainnya untuk mengembangkan, menyebarkan serta menjalankan Aplikasi, karena di dalam Oracle Application Express telah menyediakan tiga alat utama. Tiga alat utama ini memiliki Kegunaan yang penting di dalam Oracle Application Express:\\
\begin{enumerate}
\item  Application Builder.\\
Membuat aplikasi, melihat aplikasi, mengimport aplikasi, mengatur service, mengatur user aplikasi dan memantau aktifitas yang di lakukan pengguna.\\
\item  SQL Workshop.\\
Membuat tabel dan komponennya (menggunakan kode PL-SQL secara manual maupun otomatis), melihat struktur tabel dan komponennya, mengimpor dan mengekspor script.\\
\item Utilitas.\\
melihat report table dan komponennya dan history aplikasi.\\
\end{enumerate}
\paragraph{Membuat Workspace}
\begin{enumerate}
\item Pertama meng install apex terlebih dahulu.\\
\item Lalu buka link ini http://127.0.0.1:8080/apex/f?p=4050:3:13018716633464:::::\\
\item Login dengan akun yang sudah dibuat ketika meng install apex.\\
\item Jika sudah login,masuk kehalaman utama pilih creat Workspace.\\
\item Buat nama Workspace nya,lalu klik next.\\
\item Setelah itu mengubah schema nya menjadi HR,lalu klik next.\\
\item Lalu atur user name dan password untuk login.\\
\item Selanjutnya Mengatur email,setelah mengatur email klik next,lalu akan ada konfirmasi mengenai workspace.\\
\item Klik create workspace.\\
\item Lalu buka link ini http://127.0.0.1:8080/apex.\\
\item Masukan nama workspace nya sesuai dengan yang sudah dibuat tadi, user dan password nya juga.\\
\item Lalu anda akan disuruh merubah password nya.\\
\item Setelah merubah password nya anda akan masuk ke halaman utama.\\  
\end{enumerate}
\paragraph{App Builder}
application builder ini adalah untuk membuat applikasi yang berkaitan dengan database yang tersimpan pada skema di dalam oracle application express.\\
\paragraph{Sql Workshop}
Workshop SQL menyediakan alat yang memungkinkan untuk melihat dan mengelola objek database.\\
\paragraph{App Gallery}
Pada App Gallery ini terdapat beberapa aplikasi yang sudah di sediakan oleh oracle.\\
\paragraph{SpreadSheet}
Spreadsheet merupakan sebuah aplikasi atau program komputer yang digunakan untuk memanipulasi, menangkap, dan menampilkan data yang disusun di kolom dan di dalam baris. Spreadsheet juga merupakan sebuah lembaran kertas yang berisi data dalam bentuk baris dan kolom dalam akuntansi.\\
\end{document}