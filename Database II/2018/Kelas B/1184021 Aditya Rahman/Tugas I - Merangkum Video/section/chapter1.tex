\section*{Introduction Resume Seminar Oracle Database}
\paragraph{•} Pada seminar kemarin disaksikan lewat streaming, ada hal yang dapat dipahami diantaranya.
\begin{enumerate}
\item
Agenda pembelajaran \\
Pada hal ini membahas tentang aplikasi oracle express yang dimana pembahasan ini khusus tentang membuat suatu workspace pada aplikasi oracle express. \\ \\ Oracle express sendiri merupakan suatu platform pembelajaran dan perancangan (development) suatu low-code yang memperbolehkan untuk sebuah aplikasi yang aman dan terukur, dengan fitur kelas dunia yang dapat dikerjakan dimana aja. \\ 
Low-code merupakan arti dari sebuah framework yang dapat mempermudah dalam proses pengembangan aplikasi. \\
Aplikasi apex ini sangat cocok digunakan pada :
\begin{enumerate}
\item Aplikasi skala besar dengan ribuan pengguna
\item Mengisi kekosongan pada sistem perusahaan
\item Meremajakan atau memodernisasikan sistem pendahulunya
\item Aplikasi self-servicee yang untuk semua pekerja
\item Aplikasi renponsif yang dapat bekerja disemua aplikasi
\item Mengganti pengguna speradsheet
\end{enumerate}
Ada 3 alat utama yang digunakannya dalam aplikasi apex diantanya :
\begin{enumerate}
\item Aplication Builder
\\ Membuat aplikasi, melihat aplikasi, mengimport aplikasi, mengatur ser-vice, mengatur user aplikasi dan memantau aktifitas yang di lakukan peng-guna.
\item SQL Workshop
\\ Membuat tabel dan komponennya (menggunakan kode PL-SQL secaramanual  maupun  otomatis),  melihat  struktur  tabel  dan  komponennya, mengimpor dan mengekspor script.
\item Utility
\\ Melihat report table dan komponennya dan history aplikasi.
\end{enumerate}
\item Agenda pengembangan 
\begin{enumerate}
\item \textbf{local}
\begin{itemize}
\item Instal pada pc/laptop oracle express edition (XE) atau full version database
\item upgrade lah sesuai dengan  versi apex yang dibutuhkan 
\item Bisa di kerjakan pada saat offline.
\end{itemize}

\item \textbf{On-premise}
\begin{itemize}
\item dioperasikan pada bagian IT perusahaan tersebut.
\end{itemize}

\item \textbf{cloud}
\begin{itemize}
\item meluncurlkan aplikasi internet
\end{itemize}
\end{enumerate}

\item Kegunaan Aplikasi
\begin{enumerate}
\item \textbf{Apex untuk pelajar}
\begin{itemize}
\item memahami pembelajaran mengenai sql dan relasi pada database
\item praktik pengajaran SQl
\item mempelajaran aplikasi develop
\item proyek akademi
\end{itemize}

\item \textbf{Apex untuk Pengajar}
\begin{itemize}
\item penggunaan sql dan relation database
\item pengajaran menggunaan sql
\item menbangun atau mendevelopment aplikasi 
\item menvisualisikan data
\item membangun atau mendevelopment aplikasi menggunakan framework low-code
\end{itemize}
\end{enumerate}



\item Penggunaan aplikasi \\
Pengggunaan aplikasi yang dimaksud adalah bagaimana cara menggunakan aplikasi oracle expresss tersebut, pada pembahasan ini  membuat speadsheet. Tata cara dan langkah-langkahnya yaitu :
\begin{enumerate}
\item Masuk pada laman apex.oracle.com
\item Login ke workspace yang sudah dibuat.
\item klik app builder
\item klik creat new app
\item klik from file, tujuannya agar data dari speadsheet yang sudah dibuat dapat diimport kedalam workspace
\item klik copy and paste
\item pilih project and task sebagi sample set data
\item klik next
\item Masukan nama tabel, dalam kasus nya menggunakan ( speadsheet ) 
\item klik load data
\item cek pemberitahuan bahwa 73 baris sudah dimuat, jika sudah klik continue to creat aplication wizard
\item masukan nama aplikasi yang sudah dibuat
\item disebelah features klik check all
\item klik creat aplikasi
\item aplikasi yang sudah dimuat akan ditampilkan di page designer
\item klik run aplication
 
\end{enumerate}

\end{enumerate}
