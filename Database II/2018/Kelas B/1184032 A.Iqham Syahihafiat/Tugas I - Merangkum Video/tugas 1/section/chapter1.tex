\chapter{Oracle Aplication Express (APEX) }

\section{Pengenalan Oracle APEX}
Oracle Aplication Express\cite{OracleApex}, adalah aplikasi yang digunakan untuk membangun database yang dikembangkan oleh Oracle atau biasa disebut juga HTML-DB (sementara ini sampai versi terbaru masih dedicated untuk Oracle DB. Oracle APEX Adalah sebuah wadah dan sarana untuk membuat aplikasi yang menggunakan database Oracle Itu sendirMengekspor aplikasi dari Application Express sangat mudah dan menghasilkan file skrip yang dapat dibaca dengan ekstensi .SQL. Script SQL ini dapat dijalankan di lingkungan Oracle Application Express mana pun yang merupakan rilis yang sama atau lebih tinggi dari Application Express dari tempat itu diekspor. Misalnya, aplikasi yang diekspor dari Application Express 4.0 dapat diimpor ke lingkungan yang menjalankan Application Express 4.0, 4.1, atau 4.2. Namun, aplikasi yang diekspor dari Application Express 4.2 tidak dapat diimpor ke lingkungan yang menjalankan Application Express 4.1 atau yang lebih lama. Ekspor aplikasi mencakup definisi aplikasi, objek pendukung, dan komponen bersama, termasuk plug-in. Namun, ekspor tidak termasuk gambar, file CSS, file JavaScript, dll. Yang harus dikelola secara independen. Selain itu, Application Express memungkinkan pengguna untuk merancang, mengembangkan, dan menggunakan aplikasi berbasis database yang baik dan responsif. Hanya menggunakan browser web, pengguna dapat mengembangkan dan menggunakan aplikasi profesional yang cepat dan aman untuk perangkat apa pun.

\section{Fitur-Fitur Pada Oracle Apex}
Berikut adalah fitur-fitur yang terdapat di aplikasi Oracle Experes :

\begin{enumerate}

\item[1]Application Express engine membantu kita untuk membuat aplikasi secara real time dari data yang sudah disimpan di dalam table database. Ketika anda membuat atau mengembangkan sebuah aplikasi, Oracle Application Express membuat atau memodifikasi metadata yang disimpan dalam table database. Pada saat aplikasi dijalankan, Application Express engine kemudian akan membaca metadata dan menampilkan aplikasi.

\item[2]Drag and Drop file XLS, CSV, XML, atau JSON.
Jadi fitur bisa pengguna gunakan untuk mengembangkan aplikasi yang ingin dibuat dengan oracle apex dengan cara drag and drop file berupa XLS, CSV, XML, atau JSON.
    

\item[3]Membuat tabel dalam Autonomous Database. Oracle Autonomous Database mengotomatiskan semua manajemen database, infrastruktur, pemantauan, dan tuning. Hal ini dapat mengurangi biaya admin, meskipun admin masih akan diperlukan untuk tugas-tugas seperti mengelola bagaimana aplikasi terhubung ke gudang data dan bagaimana pengembang menggunakan fitur dan fungsi dalam basis data.


\item[4]Upload data into a new table. Fitur ini memungkinkan pengguna untuk mengunggah data ke table pada database yang sudah di buat di aplikasi apex.

\item[5]Create App based on new table. tidak hanya dapat menggugah data ataupun drag and drop file dari luar, dengan membuat table baru pada apex oracle, pengguna bisa membuat aplikasi berdasarkan tabel baru yang sudah dibuat oleh pengguna  


\end{enumerate}

\section{Tahapan Pembuatan Aplikasi Oracle Apex}
Langkah pertama yang harus dilakukan adalah membuka website https://apex.oracle.com, disini kita akan mendapatkan akses untuk memasuki Oracle Apllication Express, pastikan email yang dimasukkan valid untuk membuat Workspace, berikut adalah langkah langkah pembuatan Aplikasi pada Oracle APEX :

\begin{enumerate}
\item[1]Pergi ke Website Oracle APEX, https://apex.oracle.com, lalu klik Get Start For Free.

\begin{figure}[!htbp]
    \begin{center}
    \includegraphics[scale=0.2]{figures/pict(1).png}
    \caption{\textit{Go to Website Get Start For Free.}}
    \end{center}   
    \end{figure}
    
\begin{figure}[!htbp]
\item[2]Klik Request a Free Worksace.

    \begin{center}
    \includegraphics[scale=0.2]{figures/pict(2).png}
    \caption{\textit{Request A Free Workspace.}}
    \end{center}

\item[3]Isikan data diri anda seperti nama,email,dan workspace.

        
\item[4]Centang apakah anda pernah melakukan hal tersebut lalu next.  

      
\item[5]Isikan pada kolom tersebut bebas, mengapa anda ingin menggunakan layanan ini ?, lalu klik next.

 
\item[6] Centang Accept, lalu klik next.


\item[7] Tahapan terakhir untuk mengkonfirmasi apakah ini anda, lalu klik next.

   
\item[8] Workspace Sukses Dibuat dan Cek Email.

    \begin{center}
\includegraphics[scale=0.5]{figures/pict(3).jpg}
    \caption{\textit{Finish lalu Cek Email}}
        \end{center}
\label{gambar}
\end{figure}

\begin{figure}
\item[9] Workspace yang dibuat telah di Acc lalu klik continue.

    \begin{center}
\includegraphics[scale=0.2]{figures/pict(4).png}
    \caption{\textit{Email Acc.}}
        \end{center}
\label{gambar}
\end{figure}

\begin{figure}
\item[10] Workspace baru telah dibuat setelah itu lanjutkan dengan klik sign in.

    \begin{center}
\includegraphics[scale=0.5]{figures/pict(5).jpg}
    \caption{\textit{Sukses lalu Cek Email}}
        \end{center}
\label{gambar}
\end{figure}

\begin{figure}
\item[11] Sign in akun workspace yang baru saja di buat.

    \begin{center}
\includegraphics[scale=0.5]{figures/apex.jpg}
    \caption{\textit{Sign in oracle.}}
        \end{center}
\label{gambar}
\end{figure}

\begin{figure}
\item[12] sekarang kita masuk ke tampilan halaman awal aplikasi Oracle Express.

    \begin{center}
\includegraphics[scale=0.4]{figures/pict(7).jpg}
    \caption{\textit{Oracle Apex Home.}}
        \end{center}
\label{gambar}
\end{figure}

\begin{figure}
\item[13] Buka App Builder lalu klik Create New App.

    \begin{center}
\includegraphics[scale=0.4]{figures/pict(8).jpg}
    \caption{\textit{Oracle Apex App Builder}}
        \end{center}
\label{gambar}
\end{figure}

\begin{figure}
\item[14] Klik Copy and Paste, pada select data CSV atau Sample Data Set pilih Project and Tables lalu next.

    \begin{center}
\includegraphics[scale=0.2]{figures/pict(9).png}
    \caption{\textit{Sample Data Set/ Select data CSV}}
        \end{center}
\label{gambar}
\end{figure}

\begin{figure}
\item[15] Setelah Sudah me-Load data, tampilan selanjutnya akan seperti berikut. masukkan nama tabel {SPREADSHEET}, table owner, error Table Name dan Primary Keys, lalu klik load data

    \begin{center}
\includegraphics[scale=0.4]{figures/pict(10).jpg}
    \caption{\textit{Load Data.}}
        \end{center}
\label{gambar}
\end{figure}

\begin{figure}
    \begin{center}
\includegraphics[scale=0.2]{figures/pict(11).png}
    \caption{\textit{Load Data 2}}
        \end{center}
\label{gambar}
\end{figure}

\begin{figure}
\item[16] Load Data Sukses , klik Continue to Create Aplication Wizard.

    \begin{center}
\includegraphics[scale=0.2]{figures/pict(12).png}
    \caption{\textit{Oracle Apex Load Data Success.}}
        \end{center}
\label{gambar}
\end{figure}

\begin{figure}
\item[17] Sekarang kita berada di tampilan Create an Aplication, ikuti langkah berikut, buat nama App from a Spreadsheet lalu pada Features klik Check All.

    \begin{center}
\includegraphics[scale=0.4]{figures/create1.jpg}
    \caption{\textit{Create an Aplication.}}
        \end{center}
\label{gambar}
\end{figure}


\begin{figure}
\item[18]Scroll ke bawah lalu klik Create Application .

    \begin{center}
\includegraphics[scale=0.4]{figures/create2.jpg}
    \caption{\textit{Create an Application}}
        \end{center}
\label{gambar}
\end{figure}

\begin{figure}
\item[19]Tunggu beberapa saat selamat data sedang dimuat.

    \begin{center}
\includegraphics[scale=0.4]{figures/create3.jpg}
    \caption{\textit{Loading Data}}
        \end{center}
\label{gambar}
\end{figure}

\begin{figure}
\item[20]Sekarang kita masuk ke halaman App Builder project Spreadsheet yang telah berhasil dibuat. aplikasi yang baru dibuat akan tampil di halaman designer, setelah itu klik Run application .

    \begin{center}
\includegraphics[scale=0.4]{figures/create4.jpg}
    \caption{\textit{App Builder Success}}
        \end{center}
\label{gambar}
\end{figure}

\begin{figure}
\item[21]Login ke Aplikasi yang baru dibuat tadi dengan menggunakan login Oracle APEX .

    \begin{center}
\includegraphics[scale=0.4]{figures/create5.jpg}
    \caption{\textit{Sign In Spreadsheet}}
        \end{center}
\label{gambar}
\end{figure}

\begin{figure}
\item[22]Sekarang aplikasi Oracle Apex sudah dijalankan.

    \begin{center}
\includegraphics[scale=0.4]{figures/congratz.jpg}
    \caption{\textit{Welcome Spreadsheet}}
        \end{center}
\label{gambar}
\end{figure}

\end{enumerate}
