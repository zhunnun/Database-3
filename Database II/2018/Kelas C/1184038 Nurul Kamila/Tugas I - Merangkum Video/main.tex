\documentclass{article}
\usepackage[utf8]{inputenc}

\title{Resume Oracle1}
\author{nurulkamila1899 }
\date{October 2019}

\begin{document}

\maketitle

\section{Oracle APEX}
Oracle Apex Adalah Aplikais Yang Digunakan Oleh Pelanggan Nyata Untuk Aplikasi Nyata Yang Digunakan Untuk Aplikasi Kritis Oportuninistik Dan Misi Yang Melayani Puluhan Ribu Pengguna
Produk Mapan Pertama Kali Dirilis Pada Tahun 2004\\

Platform Pengembangan Aplikasi Kode Rendah Yang Paling Kuat: Memungkinkan Pengembangan Untuk Ficus Dalam Memecahkan Masalah Bisnis Dan Memberikan Solusi Yang Unggul, Dengan Lebih Sedikit Waktu Dan Upaya Yang Dihabiskan Untuk Pengodean Tingkat Rendah Biasa Dan Berulang
Terus Berkembang. Oracle Install Base Mengadopsi Oracle Apex Untuk Meningkatkan Jumlah Proyek Dan Semakin Menjadi Standar IT Korporat Yang Disetujui\\

Oracle Apex: Pengembangan Aplikasi Data Data Rendah Pertama\\
1.Kembangkan Basis Data\\
2.Mengembangkan Aplikasi\\
3.Menyebarkan\\

	Apa Itu Oracle Apex?\\
1.Mengembangkan Aplikasi Web Desktop Dan Seluler, \\
2.Memvisualisasikan Dan Memelihara Data Basis Data, \\
3.Meningkatkan Keterampilan Sql Dan Kemampuan Basis Data.\\

Oracle Apex: Use Cases\\
\subsection{	Fitur}
1.Drag Dan Letakkan File Xls, Csv, Xml, Atau Json\\
2. Membuat Tabel Dalam Database Otonom\\
3. Unggah Data Ke Tabel Baru\\
4. Buat Aplikasi Berdasarkan Tabel Baru\\
5.	Meluas Erps Dan Perangkat Lunak Perusahaan Lainnya.\\
6.	Menyediakan Dasbor Khusus Organisasi.\\
7.	Alur Kerja Yang Lebih Baik.\\
8.	Mengisi Kekosongan.\\
9.	Fitur Tanpa Biaya Dari Database Oracle\\

\subsection{Fitur Didukung Penuh Tanpa Biaya}
1.	Banyak Aplikasi, Pengembang & Pengguna Akhir\\
2.	Tim Dukungan Oracle Khusus\\
3.	11gr2, 12c, 18c\\
4.	Semua Edisi DB EE, SE2, XE , Termasuk Dengan Layanan Clound Oracle\\

\subsection{Database Otomatis}
1.	Database Sebagai Layanan\\
2.	Tidak Ada Evaluasi Biaya Http: //Apex.Oracle.Con\\

\subsection{Mudah Dipasang}
1.	Termasuk Secara Default Dengan Semua Edisi Database Oracle\\
2.	Download Rilis Terbaru Dari Https://Apex.Oracle.Com\\

\subsection{Solusi}
1.	Sumber Tunggal Kebenaran\\
2.	Kirim Url Bukan File\\
3.	Aplikasi Aman, Terukur, Multi-Pengguna\\
4.	Diperluas Dengan Bagan, Kalender, Validasi, Dan Banyak Lagi\\
5.	 Memenuhi Persyaratan Non-Standar\\
6.	 Mengoptimalkan Fungsi Bisnis Umum\\
7.	 Meningkatkan Pengambilan Data\\
8.	Integrasikan Sumber Data Yang Berbeda\\

\subsection{Membedakan}
Ciri:\\
1. Ide Pengembangan App Adalah Browser Web. Tidak Ada Perangkat Lunak Klien Yang Dibutuhkan\\
2. Devinitions Disimpan Dalam Database Sebagai Metadata.Declarative Tidak Ada Generasi Kode\\
3. Generasi Halaman Adalah Efisiensi Dengan Hanya Satu Permintaan Dan Satu Respons.\\
4. Puncak Oracle Pada Database Otonomi Oracle : Mengembangkan, Menyesuaikan, Dan Mengirimkan Dengan Cepat.\\
5. Kontrol Pre-Built Untuk Keamanan, Autentikasi, Interaksi Database, Validasi, Manajemen Sestion Dan Banyak Lagi.\\
6. Beralih Dari Prototipe Ke Produksi Dalam Hitungan Menit.\\
7. Asitectur Apex\\
8. Single Row(Http)\\
9. Multi Row(Jdbc)\\

\subsection{Contoh Database Tunggal / Beberapa Ruang Kerja.}
1.	Ruang Kerja Yang Digunakan Untuk Definisi / Skema.\\
2.	Administrator Contoh Mengelola Lingkungan Dan Akses Skema.\\
3.	Departemen Dapat Meminta Lebih Banyak Ruang, Dan Akses Ke Skema Baru.\\

\subsection{Opsi Deveploment / Penyebaran}
\subsubsection{Lokal}
1.	Instal Pada Laptop Yang Berdiri Sendiri Menggunakan Edisi Oracle Express Edition (Xe) Atau Database Lengkap.\\
2.	Cukup Tingkatkan Apex Ke Versi Yang Diperlukan.\\
3.	Dapat Bekerja Sepenuhnya Terputus.\\

\subsubsection{Di Tempat}
1.	Biasanya Dijalankan Oleh Departemen TI\\
2.	TI Umumnya Adalah Layanan Operasi Produksi, Dan Penyedia Layanan\\
3.	Departemen Yang Bertanggung Jawab Untuk Pengembangan Aplikasi.\\
4.	Fitur Tanpa Biaya Dari Database Oracle\\

\subsection{Fitur Didukung Penuh Tanpa Biaya}
1.	Banyak Aplikasi, Pengembang & Pengguna Akhir\\
2.	Tim Dukungan Oracle Khusus\\
3.	11gr2, 12c, 18c\\
4.	Semua Edisi DB EE, SE2, XE\\

\subsection{Clound}
1.	Aplikasi Internet -Deploy\\
2.	Laveraged Untuk Pengembangan Aplikasi Yang Cepat, Penerimaan Pengguna Dan Pelatihan\\
3.	Protipe & Bukti Konsep\\
4.	Perusahaan Konsultasi Mengembangkan Untuk Penempatan Di Tempat Pelanggan.\\

\subsection{Pendidikan}
Apakah Anda Seorang Siswa Atau Guru SQL, Database Relasional, atau Pengembangan Aplikasi, Anda Dapat Menggunakan Oracle Apex Untuk Sangat Memperkaya Pengalaman Pendidikan Anda?\\

\subsection{Sertifikasi APEX}
Setelah Anda Mahir Mengembangkan Aplikasi APEX, Anda Dapat Mengikuti Ujian Sertifikasi Oracle Menjadi Aplikasi Oracle Express 18: Profesional Bersertifikat Pengembang.\\

Menonjol Di Antara Rekan-Rekan Anda, Dan Buktikan Kepada Semua Orang Bahwa Anda Tahu Cara Membangun Aplikasi Yang Kuat Dengan Menggunakan Apex.\\

\subsection{Kurikulum Gratisan Oracle Apex}
1.	Pelajar, Dan Panduan Praktikum Di Laboratorium\\
2.	Total 16 Pelajaran Dan 15 Tangan Di Laboratorium\\
3.	PPT, PDF, Sumber, Dan File Lab\\
4.	Lab / Demo Dapat Dilakukan Pada: Contoh Akademi Oracle\\

\subsection{Gambaran Umum }
Lab Ini Menuntun Anda Saat Mengunggah Spreadsheet Ke Tabel Database Oracle, Lalu Membuat Aplikasi Berdasarkan Tabel Baru Ini.  Anda Kemudian Akan Bermain Dengan Laporan Interaktif Dan Meningkatkan Formulir Terlampir.  Terakhir, Anda Akan Menambahkan Halaman Kalender Dan Kemudian Menautkannya Ke Halaman Formulir Yang Ada.  Alih-Alih Mencoba Mengirim Surel Spreadsheet Untuk Mengumpulkan Informasi Dari Orang Yang Berbeda, Cukup Buat Aplikasi Dalam Hitungan Menit, Dan Kirim Surel URL.  Spreadsheet Sumber-Kebenaran-Tunggal, Multi-Pengguna, Aman, Dan Mudah Tersiram Ini!  Aplikasi Scren Jadi Lebih Baik\\

\subsubsection{Langkah 2.1 - Masuk}
1.	Masuk Ke Ruang Kerja Anda Di Apex.Oracle.Com\\
2.	Klik Pembuat Aplikasi\\
3.	Klik Buat Aplikasi Baru\\
\subsubsection{Langkah 2.2 - Memilih Jenis Aplikasi}
1.	Klik Dari File\\
\subsubsection{Langkah 2.3 - Memuat Data Sampel}
1.	Klik Salin Dan Tempel\\
2.	Klik Selanjutnya\\
\subsubsection{Langkah 2.6 - Memberi Nama Aplikasi}
1.	Nama Enter {App From A Spreadsheet\\
2.	Berikutnya Ke Fitur, Klik Centang Semua\\
\subsubsection{Langkah 2.7- Buat Aplikasi}
1.	Klik Buat Aplikasi
\subsubsection{Langkah 2.8- App In Page Desaigner}
1.	Di Aplikasi Baru Kamu Akan Ditampilkan Desainer\\
2.	Klik Run Aplikasi
\subsubsection{Langkah 2.9- Runrime Aplikasi}
1.	Masukkan Kredensial Pengguna Anda\\
2.	Bermain-Main Dengan Aplikasi Baru Anda
\subsubsection{Langkah 3.1-  Urutkan Laporan Interaktif}
1.	Klik Spreadsheet\\
2.	Clik Actions, Select Data, Select Sort\\
3.	Untuk 1, Select Start Datte; Untuk 2, Select End Date; Clik Apply\\
4.	Menggunakan Lingkungan Runtime\\
5.	Memperbaiki Laporan Dan Formulir\\
\subsubsection{Langkah 3.2- Menambahkan Komputasi}
1.	Klik Actions, Pilih Data, Pilih Compute\\
2.	Column Label Masuk Bugget V Cost\\
3.	Format Maskpilih $5,243,10\\
4.	Masukkan Ekspresi Komputasi I-H\\
5.	Klik Apply
\subsubsection{Langkah 3.3 “Menambahkan Sebuah Grafik"}
1.	Klik “Action”, Dan Pilih “Chart”\\
2.	Label Pilih “**Budget V Cost”\\
3.	Fungsi Pilih “ Sum”\\
4.	Sort Pilih “Label-Ascending”\\
5.	Orientasi Pilih “Horizontal”\\
6.	Klik “Apply”\\
*NB: Untuk Pengeditan Grafik Bias Dilakukan Dilaman App From Spreadsheet
\subsubsection{Langkah 3.4 “Menyimpan Laporan”}
1.	Klik”Action”,Pilih “Report”, Pilih”Save Report”\\
2.	Untuk Simpan, Pilih “As Default Report Settings”\\
3.	Tipe Default Laporan, Pilih “ Alternative”\\
4.	Nama, Enter “Data Review”\\
5.	Klik “Apply”\\
\subsubsection{Langkah 3.5 “Batasi Status”}
1.	Ketika Runtime Environment, Klik “Edit Icon On A Record”\\
2.	Halaman Modal Akan Tampil\\
3.	Pada Developer Toolbar, Klik “Quick Edit”\\
4.	Pada Status Item (Tunggu Sampai Outline Biru Muncul” Lalu Klik Mouse\\
5.	Di Halaman Designer Muncul Dengan Focus Pada Status Item\\
6.	Di Halaman Designer, Dalam Editor Property(Panel Kanan)\\
7.	Di Bawah Daftar Nilai-Nilai, Untuk Type Pilih “SQL Query”\\
8.	Lanjutkan Ke SQL Query, Klik “Code Editor”\\
\subsubsection{
Langkah 3.5 C
Membatasi Status (Restrict The Status)}
1.	Dalam Kode Masukkan Seperti Ini\\
2.	Select Distinct Status D, Status R From Spreadsheet Order By 1\\
3.	Klik Validate\\
4.	Klik Ok\\
5.	Untuk Menampilkan Nilai Ekstra Pilih No\\
6.	Menampilkan Nilai Null Pilih –Select Status\\
7.	Klik Save (Pada Toolbar Top Right)
\subsubsection{Langkah 3.6
Menjalankan Apllikasi (Run Apllikasi)}
1.	Arahkan Navigasi Ke Runtime Environment\\
2.	Refresh Browser\\
3.	Edit Record\\
4.	Klik Status
\subsubsection{Langkah 4.1 Tambah Kalender (Add A Calender)}
1.	Arahkan Navigasi Kembali Ke Development Environtment\\
2.	Pada Aplikasi Builder, Arahkan Pada Home Page\\
3.	Klik Create Page
\subsubsection{Langkah 4.2 - Menautkan Kalender Ke Dari}
1.	Di Tab Rendering, Di Bawah Kalender, Klik Atribut\\
2.	Di Editor Properti (Panel Kanan), Klik Lihat / Edit Tautan\\
3.	Halaman, Pilih 3\\
4.	Mengatur Item-Nama, Pilih P3_ID; Nilai, Pilih ID\\
5.	Bersihkan Cache, Masukkan 3\\
6.	Klik Ok\\
7.	Klik Simpan Dan Jalankan\\
\subsection{Pengertian Spreadsheet}
Spreadsheet: Memungkinkan Pengguna Untuk Menyimpan Berbagai Informasi Yang Sangat Lengkap, Pada Setiap Kolomnya Bisa Menyimpan Berbagai Data Informasi Yang Berbeda Dari Informasi Yang Di Perlukan.\\

App From Spreadsheet Disini Berupa Beberapa Project Dan Nama Tugas Nya Serta Keterangan Lainnya Seperti Tanggal Mulai, Tanggal Selesai, Status, Di Ttd Oleh,Biaya, Budget Tersedia, Dan Lebih Kurangnya Dari Budget.
\subsection{Membuat Aplikasi Dari Spreadsheet}
\subsubsection{Langkah 1.1 A}
1.Pergi Ke Http://Apex.Oracle.Com\\
2.Klik Get Started For Free

\subsubsection{Langkah 1.1b}
1.	Klik Permintaan Ruang Kerja Yang Kosong.\\

\subsubsection{Langkah 2.1 Masuk}
1. Masuk Ke Ruang Kerja Anda Di Http://Apex.Oracle.Com\\
2.Klik Pembuat Aplikasi Klik Buat Aplikasi Baru

\subsubsection{Latihan 2.5 Memverifikasi Catatan Dimuat}
1. Centang Bahwa 73 Baris Dimuat\\
2. Klik Terus Untuk Membuat Aplikasi Wizard


\subsubsection{Tautan Yang Bermanfaat}
1.	Apex  Collateral Http://Apex.Oracle.Com\\
2.	2.Tutorial Http://Apex.Oracle.Com/En/Learn/Tutorial\\
3.	Community External Site + Slack Http://Apex.Oracle.Com/Community

\subsubsection{Langkah 4.2b - Menautkan Kalender Ke Formulir Pembaruan}
Note: Anda Mungkin Harus Menavigasi Ke Bulan Mei Untuk Melihat Entri Kalender

\subsection{Jenis Aplikasi Apa Yang Cocok Untuk Oracle Apex?}
1. Aplikasi Kritis Yang Besar Untuk Ribuan Pengguna\\
2. Mengisi Kesenjangan Dalam Sistem Perusahaan\\
3. Streamline Proses Bisnis Yang Sudah Ketinggalan Zaman\\
4. Modernisasi Sistem Warisan\\
5. Aplikasi Swalayan Untuk Semua Karyawan\\
6. Portal Pelanggan / Mitra Menghadap\\
7. Aplikasi Responsif Yang Bekerja Pada Perangkat Apa Pun\\
8. Bukti Konsep\\
9. Aplikasi Cepat-Menang (Umur <Beberapa Bulan)\\
10.Mengganti Spreadsheet

\subsection{Mengelola Data Dalam Spreadsheet Itu Sulit}
1.	Memvalidasi Data Secara Manual Dan Rawan Kesalahan\\
2.	Integrase Data -Tidak Dapat Menjamin Keakuratan Data Di Lingkungan Multi-Pengguna.\\
3.	Penguncian Sel Keamanan Data Tidak Efektif.\\
4.	Berbagi Data-Excel Lamban Dan Sulit Untuk Dibagikan\\
5.	Puncak Oracle

\subsection{Agenda}
1. Pengembangan Aplikasi Kode Rendah\\
2. Aplikasi Oror Express: Ikhtisar\\
3.Mengonversi Spreadsheet Ke Aplikasi Web Dalam Hitungan Menit- Demo Aplikasi \\
4.Oracle Mengungkapkan Fitur Produk Dan Demo\\
5.Oracle Apex: Pendidikan\\
6. Laboratorium Tangan

\subsection{Apa Kode Rendah?}
1.Mudah Di Jalan\\
2.Sangat Produktif\\
3.Scalable\\
4.Dapat Diperpanjang\\
5.Fungsionalitas Yang Kaya Dengan Kode Yang Lebih Sedikit.

\subsection{Pengembangan Aplikasi Di Perusahaan.}
1.Membutuhkan Sumber Daya Pengembangan Khusus Dan Mahal\\
2.Panjang Siklus Dev Aplikasi\\
3. Backlog Yang Besar\\
4. Kolaborasi Minimal\\
5.Bisnis Menyelesaikan Masalah Dengan Alat Yang Salah


\subsection{Cara Pengembangan Aplikasi Data Rendah Pertama}
1.Mulai\\
2.Pengadaan Lalu Masuk Ke Dalam Database\\
3.Setelah Itu Diadakan Pengadaan Lagi\\
4.Pada Tahap Selanjutnya Adalah Tahap Pembangun Aplikasi\\
5.Lalu Aplikasi Yang Telah Dibangun Akan Dilakukan Pengetesan Dan Timbal Balik Ke Aplikasi Produk Yang Dibangun Lalu Masuk Lagi Ke Dalam Database Sql\\
6. Setelah Itu Akan Dilakukan Install Dan Pembaharuan\\
7 Setelah Itu Akan Dikirim Ke Aplikasi Pembangunan.

\subsection{Cara Membuat Database Baru}
1.New\\
2.Wizard\\
3.Blueprint

\subsection{Cara Pengembangan Aplikasi}
1.New\\
2.Markdown\\
3.Model\\
4.Sql Ide\\
5.Dml Script\\
6.Excel




































\end{document}
