\documentclass [12pt, times new roman, a4paper]{article}
\usepackage[utf8]{inputenc}
\title{RESUME ORACLE EXPRESS (APEX)}
\author{ \textbf{Alvian Daniel Sinaga 1184077}}

\date{October 2019}
\begin{document}
\maketitle
\paragraph{Oracle APEX}Adalah platform pengembangan kode sangat rendah yang bukan hanya memungkinkan Anda membangun aplikasi perusahaan dan perorangan yang memiliki lingkup tertentu dan aman dengan fitur kelas dunia yang dapat digunakan di mana saja, dalam perangkat lunak apapun.

\section{Aplikasi Yang Dapat digunakan Dengan Apex Oracle}
1. large mission-critical apps for thousands of user\\
2. Fill in gaps in corporate system\\
3. Modernization of legacy system\\
4. Streamline outdated business processes\\
5. self-service apps for all employees\\
6. customer/ partner-facing portals\\
7. Responsive apps that work on any device\\
8. proof of concepts\\
9. Quick-win apps(lifespan < a few months)\\
10. Replacing spreadsheets.
\section{Cara Pengembangan Aplikasi Data Rendah
Pertama}
1. Mulai\\
2. Pengadaan Lalu Masuk Ke Dalam Database\\
3. Setelah Itu Diadakan Pengadaan Lagi\\
4. Pada Tahap Selanjutnya Adalah Tahap Pembangun Aplikasi\\
5. Lalu Aplikasi Yang Telah Dibangun Akan Dilakukan Pengetesan Dan\\ Timbal Balik Ke Aplikasi Produk Yang Dibangun Lalu Masuk Lagi Ke Dalam Database Sql\\
6. Setelah Itu Akan Dilakukan Install Dan Pembaharuan\\
7. Setelah Itu Akan Dikirim Ke Aplikasi Pembangunan.
\section{Penggantian Spreadsheet}
\subsection{Fitur}
1. Seret dan Jatuhkan file XLS, CSV, XML, atau JSON\\
2. Buat tabel dalam Database Autonomous\\
3. Unggah data ke tabel baru\\
4. Buat Aplikasi berdasarkan tabel baru\\

\subsection{Solusi}
1. Sumber tunggal kebenaran\\
2. mengirimkan URL bukan file\\
3. Aplikasi yang aman, terukur, multi-pengguna\\
4. Perluas dengan bagan, kalender, validasi, dan lainnya\\
\section{Memodernisasi Formulir Oracle}
\subsection{Fitur}
1. APEX adalah evolusi alami bentuk\\
2. Keduanya berdasarkan SQL dan PL / SQL\\
3. Gunakan kembali paket DB, prosedur, Fungsi\\
4. Mudah melatih Pengembang untuk berkembang dengan APEX\\
 
\subsection{Solusi}
1.Proof-of-concept menggunakan subset dari aplikasi formulir\\
2. Aplikasi di seluruh organisasi, bukan di kantor\\
3. Aplikasi eksternal untuk pelanggan / mitra\\
4. Aplikasi mobile-first\\
5. Persyaratan baru bersih\\
\section{Bangun Aplikasi Besar, Skala Perusahaan}
\subsection{Fitur}
1. Quickly build custom mission-critical apps\\
2. Corporate data reporting and maintenance\\
3. Merge separate data silos\\
4. Build organization-wide self-reporting apps\\
\subsection{Solusi}
1. penyebaran sederhana\\
2. Instalasi tunggal menjalankan sejumlah aplikasi\\ 
3. Skala ke ratusan ribu pengguna\\
4. Memanfaatkan fitur perusahaan dari Oracle Autonomous\\
\section{Fitur Tanpa Biaya dari Database Oracle}
\subsection{Fitur Didukung Penuh tanpa Biaya}
1. Sejumlah aplikasi, pengembang dan pengguna akhir\\
2. Tim Dukungan Oracle khusus\\
3. 11gR2,12c, 18c\\
4. Semua edisi DB: EE, SE2, XE\\
\subsection{Termasuk dengan Layanan Oracle Cloud}
1. Basis Data Otonom\\
2. Basis data sebagai layanan\\
3. Tanpa biaya evaluasi http://apex.oracle.com\\
\subsection{Mudah dipasang}
1. disertakan secara default dengan semua edisi database Oracle\\
2. Unduh rilis terbaru dari http://apex.oracle.com/otp\\
\section{Berkembang dengan Cepat, Menyesuaikan, dan Memberikan}
Kontrol yang dibuat sebelumnya untuk keamanan, otentikasi, interaksi basis data, validasi, manajemen sesi, dan lainnya. beralih dari prototipe ke produksi dalam hitungan menit
\section{Opsi Pengembangan / penyebaran}
\subsection{Lokal}
1. Instal pada laptop yang berdiri sendiri menggunakan Oracle Express Edition (XE) atau versi database lengkap\\
2. cukup tingkatkan APEX ke versi yang diperlukan\\
3. Dapat bekerja sepenuhnya terputus\\
\subsection{On-Premis}
1. Biasanya dijalankan oleh Departemen TI\\
2. TI umumnya adalah layanan operasi produksi, dan penyedia layanan\\
3. Departemen yang bertanggung jawab untuk pengembangan aplikasi\\
\subsection{Cloud}
1. Menyebarkan aplikasi Internet\\
2. Ditingkatkan untuk penerimaan dan pelatihan pengguna\\ pengembangan aplikasi yang cepat.\\
3. Prototyping dan Proof-of-concept\\
4. Perusahaan konsultan berkembang untuk ditempatkan di lokasi pelanggan\\
\subsection{Mesin Virtual Basis Data Tunggal / Beberapa Ruang Kerja}
1. Ruang kerja yang digunakan untuk mendefinisikan definisi aplikasi / Skema menyimpan data\\
2. Hubungan many to many antara ruang kerja dan skema\\
3. Instance Administrator mengelola lingkungan dan akses skema\\
4. Departemen dapat meminta lebih banyak ruang, dan akses ke skema baru\\
\section{Community}
1. Lebih dari 500,00 pengembang di seluruh dunia\\
2. Lebih dari 100 blogger aktif\\
3. Perusahaan konsultan, buku, kesuksesan, cerita, kutipan, aplikasi komersial\\
\section{Oracle APEX untuk Guru}
1. SQL dan Database Relasional\\
2. Praktek di laboratorium dengan SQL\\
3. Pengembangan aplikasi\\
4. Aplikasi kode rendah dev\\
5. Visualisasi Data\\
6. Memulai secara gratis di https://apex.oracle.com atau gunakan contoh Oracle Academy APEX\\
7. Menyiapkan akun siswa\\
\section{Oracle APEX Kurikulum Gratis}
1. Pelajaran, dan Panduan praktik di laboratorium.\\
2. Total 16 Pelajaran dan 15 Praktek di laboratorium\\
3. PPT, PDF, sumber dan file lab\\
4. Laboratorium / Demo dapat dilakukan pada:\\
-https: //apex.oracle.com\\
-https: //cloud.oracle.com/try-autonomous-database\\
-Oracle Academy Instance (setelah naik ke APEX 19.1)
\section{Memperoleh Ruang Kerja}
\textbf{Step 1.1A}\\
1. Pergi ke http://apex.oracle.com\\
2. klik mulai secara gratis\\
\textbf{Step 1.1b}\\
1. klik permintaan workspace\\
/\textbf{Step 1.2}
1. Apa jenis ruang kerja-Klik Pengembangan Aplikasi\\
2. Masukkan rincian Identifikasi Anda-Nama depan, Nama Belakang. Email, Ruang Kerja\\
3. Masukkan detail Skema-Nama Skem
4. Lengkapi wizad\\
\textbf{Step 1.3}\\
1. Periksa email Anda, Anda harus mendapatkan email dari oracle application-express@oracle.com dalam beberapa menit\\
2. klik Buat Ruang Kerja\\
3. Klik Lanjutkan ke layar masuk\\
4. Setel ulang kata sandi Anda\\
\textbf{Step 2.1}
1. log into your workspace on http://apex.oracle.com\\
2. Click App Builder\\
3. Click Create a New app\\
\textbf{Step 2.2}
1. masuk ke ruang kerja Anda di http: //apex.oracle.com \\
2. Klik Pembuat Aplikasi \\
3. Klik Buat aplikasi Baru \\
\textbf{Jenis Aplikasi Slecting} 
1. Klik Formulir File \\
\textbf{Langkah 2.3} \\
1. Klik Salin dan Tempel \\
2. Untuk Sampel Data Set pilih Proyek dan Tugas \\
3. Klik Berikutnya \\
\textbf{Langkah 2.4 Memberi nama tabel} \\
1. Masukkan Nama Tabel {SPREADSHEET} \\
2. Klik Muat Data \\
\textbf{Langkah 2.5- Memverifikasi Reecords yang Dimuat}\\
1. Periksa apakah 73 baris sudah dimuat \\
2. Klik Lanjutkan untuk Membuat Wisaya Aplikasil \\
\textbf{Langkah 2.6-Menamai aplikasi} \\
1. Masukkan Nama \\
2. Di sebelah fitur, klik centang Semua \\
\textbf{Langkah 2.7- Buat Aplikasi} \\
1. Klik Buat Aplikasi \\
\textbf{Langkah 2.8-Aplikasi di halaman Desainer} \\
1. Aplikasi baru Anda akan ditampilkan di halaman Designer \\
2. Klik Jalankan Aplikasi \\
\textbf{Langkah 2.9-Aplikasi Runtime} \\
1. Masukkan kredensial pengguna Anda \\
2. Bermain-main dengan aplikasi baru Anda \\
\ section {Memperbaiki Laporan dan formulir}
\textbf{Langkah 3.1 - Sortir Laporan Interaktif} \\
1. Klik Spreadsheet \\
2. Klik Tindakan, Pilih Data, Pilih Sortir \\
3. Untuk 1, pilih Tanggal Mulai; Untuk 2 Tanggal Berakhir; Klik Terapkan \\
\textbf{Langkah 3.2 - Tambahkan Perhitungan} \\
1. Klik Tindakan, Pilih Hitung \\
2. Label Kolom masukkan Biaya V Biaya \\
3. Format Mask pilih  5,234.10 \\
4. Ekspresi Komputasi masukkan I - H \\
5. Klik Terapkan \\
\textbf{Langkah 3.3 - Tambahkan grafik} \\
1. Klik Tindakan, Pilih Bagan \\
2. Label pilih Project \\
3. Nilai Pilih ** Anggaran V Biaya \\
4. Fungsi pilih Jumlah \\
5. Sortir pilih Label - Ascending \\
6. Orientasi pilih Horizontal \\
7. Klik Terapkan \\
\textbf{Langkah 3.4 - Simpan Laporan} \\
1. Klik Tindakan, pilih Laporan, Pilih Simpan Laporan \\
2. Untuk Menyimpan, Pilih Sebagai Pengaturan Laporan Default \\
3. Jenis Laporan Default, pilih Alternatif \\
4. Nama, Masukkan Tanggal Ulasan \\
5. Klik Terapkan \\
\textbf{Langkah 3.5 - Batasi Status} \\
1. Di lingkungan runtime, Klik ikon edit pada catatan \\
2. Halaman modal akan ditampilkan \\
3. Di Bilah Alat Pengembang, Klik Edit Cepat \\
4. Arahkan kursor ke item Status (sampai garis besar biru appers) dan klik mouse \\
5. Desainer Halaman menampilkan dengan fokus pada item Status \\
\textbf{Langkah 3.5b - Batasi Status} \\
1. Di halaman Designer, di dalam Property Editor (panel kanan), untuk Type pilih Select list \\
2. Di bawah Daftar Nilai, untuk Ketik pilih SQL Query \\
3. Di sebelah SQL Query, klik Editor Kode \\
\textbf{Langkah 3.5c - Batasi Status} \\
1. Di dalam Editor Kode, masukkan yang berikut: Pilih status berbeda d, status r urutan spreadsheet oleh 1 \\
2. Klik Validasi \\
3. Klik OK \\
4. Tampilkan Nilai Ekstra, pilih Tidak \\
5. Tampilan Nilai Null, Masuk - Pilih Status - \\
6. Klik simpan (Di bilah alat -tepat ke kanan) \\
\textbf{Langkah 3.6 - Jalankan Aplikasi} \\
1. Navigasikan kembali ke lingkungan runtime \\
2. Refresh browser \\
3. Edit catatan \\
4. Klik Status \\
\textbf{Langkah 4.1c - Tambahkan Kalender} \\
1. Preferensi Navigasi, Klik Buat entri menu navigasi baru \\
2. Klik Berikutnya \\
3. Tabel / Lihat Nama, Pilih SPREADSHEET (tabel) \\
4. Klik Berikutnya \\
\textbf{Langkah 4.2 - Tautkan kalender ke Formulir Pembaruan} \\
1. Di tab Rendering, di bawah Kalender, Klik Atribut \\
2. Di Editor Properti (panel kanan), klik Lihat / Edit Tautan \\
3. Halaman, Pilih 3 \\
4. Setel Item - Nama, pilih P3\textunderscore ID garis bawah; Nilai, Pilih ID \\
5. Bersihkan Cache, Masukkan 3 \\
6. Klik OK \\
7. Klik Simpan dan jalankan.
\section{Oracle APEX}
Oracle Apex Adalah Aplikais Yang Digunakan Oleh Pelanggan Nyata Untuk
Aplikasi Nyata Yang Digunakan Untuk Aplikasi Kritis Oportuninistik Dan Misi
Yang Melayani Puluhan Ribu Pengguna Produk Mapan Pertama Kali Dirilis
Pada Tahun 2004 Platform Pengembangan Aplikasi Kode Rendah Yang Paling Kuat: Memungkinkan Pengembangan Untuk Ficus Dalam Memecahkan
Masalah Bisnis Dan Memberikan Solusi Yang Unggul, Dengan Lebih Sedikit
Waktu Dan Upaya Yang Dihabiskan Untuk Pengodean Tingkat Rendah Biasa
Dan Berulang Terus Berkembang. Oracle Install Base Mengadopsi Oracle Apex
Untuk Meningkatkan Jumlah Proyek Dan Semakin Menjadi Standar IT Korporat Yang Disetujui.
\section{Creating a Database with \textit{Oracle Quick SQL} Workshop}
\textbf{Langkah 1}
\begin{enumerate}
    \item Buat dua tabel termasuk hubungan di antara mereka.
    \item Masukkan datanya ke dalam tabel.
    \item Tambahkan nilai khusus ke kolom.
    \item Buat Tampilan.
    \item Konfigurasikan pengaturan.
    \item Buat dan jalankan skrip untuk membuat basis data.\\
    \end{enumerate}
\textbf{Langkah 2 \textit{Jalankan Cepat SQL dan Buat Tabel}}
\begin{enumerate}
    \item Pertama Kamu akan dibawa ke halaman masuk akun oracle.
    \item Jika kamu mempunyai akun maka lakukan masuk halaman.
    \item Jika tidak mempunyai akun , buat akun.
    \end{enumerate}
\textit{Intruction}
\begin{enumerate}
    \item 
Sekali login Anda akan melihat layar ini dan Anda akan mengetik tabel dan nama field Anda,
    \item kemudian lihat SQL yang dihasilkan.
    \item Saat Anda mengetik, SQL untuk membuat 2 tabel sedang dihasilkan. perhatikan Primary key bidang-id secara otomatis dibuat, serta foreign key student-id dalam tabel kursus.
\end{enumerate}
\textbf{Langkah 3 Tambahkan NOT NULL dan Periksa Batasan}
\textit{Intruction}
\begin{enumerate}
    \item Masukkan sytanx yang ditunjukkan untuk membuat batasan pemeriksaan pada bidang halaman.
    \item dan, kemudianvclik genereted untuk memyegarkan kode.
\end{enumerate}
\textbf{Langkah 4 Masukkan Data ke dalam Tables}
\textit{Intructions}
\begin{enumerate}
    \item Kembali ke setiap tabel dan tambahkan / sisipkan sintaks seperti yang ditunjukkan untuk menghasilkan pernyataan sisipan [catatan baru] untuk setiap tabel.
\end{enumerate}
\textbf{Langkah 5 Menambahkan nilai khusus}
\begin{enumerate}
    \item menambahkan nilai bahasa selanjutnya untuk bidang utama, seperti yang ditunjukkan untuk membatasi nilai dalam data yang dihasilkan.
\end{enumerate}
\textbf{Langkah 6 Membuat Tampilan}
\begin{enumerate}
    \item Tambahkan sintaks yang ditunjukkan untuk membuat tampilan yang disebut studentscourses termasuk semua bidang dari tabel siswa dan kursus.
\end{enumerate}
\textbf{Langkah 7 Mengatur Setinggan}
\begin{enumerate}
    \item Menampilkan DROP statements.
\end{enumerate}
\textbf{Langkah 8 Membuat dan menjalankan sebuat skrips}
\begin{enumerate}
    \item Jika Anda telah diberi apex akun dari oracle academy atau guru Anda kemudian masuk.
\end{enumerate}
\textbf{Mencoba sendiri Program}

\section{Oracle SQL Developer Data Modeler Workshop}
\begin{enumerate}
\item Memulai dengan Oracle SWQL Developer Data Modeler
\par Oracle SQL Developer Data Modeler menawarkan serangkaian data dan pemodelan basis data kapabilitas, memungkinkan Anda untuk:
\begin{enumerate}
\item Capture aturan dan informasi bisnis
\item Membuat dan memproses logical, relation, dan physical Models
\item Simpan informasi metadata dalam file XML
\item Sinkronkan Model Relasional dengan Kamus Data
\end{enumerate}
\par Konsep Kunci:
\begin{enumerate}
\item Buat Model logis menggunakan SQL Data Modeler
\item Teknisi Maju Model Logika ke Model Relasional
\item Reverse Engineer Model Relasional
\item Terapkan Standar Penamaan menggunakan:
\par - Glosarium
\par - Template Penamaan
\par\textbf{Kesulitan:} Pemula-Workshop ini cocok untuk seseorang yang belum pernah menggunakan Oracle SQL. Pengembang Pemodel Data tetapi memiliki pengetahuan dasar tentang metode dan terminologi Desain Basis Data.
\par \textbf{Durasi:} Kira-kira 2-3 jam
\end{enumerate}
\item Mengunduh Oracle SQL Developer Data Modeler
\begin{enumerate}
\item Untuk mengunduh file instalasi, buka Oracle Technology Network
\item Pastikan Anda memiliki JRE yang diinstal, jika tidak, unduh dari Oracle Technology Network
\end{enumerate}
\item Buka Oracle SQL Developer Data Modeler
\par Setelah file zip Data Modeler diunduh:
\begin{enumerate}
\item Ekstrak file zip ke folder apa pun
\item Di dalam folder itu perluas folder datamodeler
\item Klik dua kali datamodeler.exe untuk 32-bit dan klik dua kali datamodeler64.exe untuk 64-bit
\item referensi informasi nilai pada Halaman Awal (halaman ini dapat dibuka kembali dengan mengklik Bantuan, Halaman Awal)
\item Tutup Jendela Mulai
\item kami siap untuk pergi
\end{enumerate}
\item Global Fast Foods ERD
\begin{enumerate}
\item Ini adalah Global Fast Foods ERD lengkap tetapi kami akan mulai dengan versi yang lebih sederhana yang ditemukan pada slide berikutnya.
\item Buat entitas ini dalam SQL Data Modeler menggunakan petunjuk pada bilah geser berikut.
\end{enumerate}
\end{enumerate}

\end{document}