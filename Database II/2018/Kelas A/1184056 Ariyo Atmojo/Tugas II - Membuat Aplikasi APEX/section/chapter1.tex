\chapter{Oracle Aplication Express}

\section{Tahapan Pembuatan Aplikasi Oracle Apex}
Pada Langkah pertama yang harus dilakukan untuk membuat sebuah aplikasi pada Apex Oracle adalah membuka website https://apex.oracle.com, didalamnya kita akan mendapatkan akses untuk memasuki Oracle Apllication Express, pastikan email yang dimasukkan sesuai untuk membuat Workspace, Berikut di bawah ini adalah langkah - langkah pembuatan Aplikasi pada Oracle APEX :

\begin{enumerate}
\item[1]Pergi ke Website Oracle APEX, https://apex.oracle.com, lalu klik Sign In.

\begin{figure}[!htbp]
    \begin{center}
    \includegraphics[scale=0.2]{figures/13.png}
    \caption{\textit{Sign In.}}
    \end{center}   
    \end{figure}
    
\begin{figure}[!htbp]
\item[2]Lalu Masukan Workspace yang telah dibuat sebelumnya.

    \begin{center}
    \includegraphics[scale=0.2]{figures/1.png}
    \caption{\textit{Sign In Workspace.}}
    \end{center}


\item[3] Setelah Sign In, Klik App Builder .

    \begin{center}
\includegraphics[scale=0.2]{figures/2.png}
    \caption{\textit{App Builder}}
        \end{center}
\label{gambar}
\end{figure}

\begin{figure}
\item[4] Setelah klik App Builder, lalu klik Create.

    \begin{center}
\includegraphics[scale=0.2]{figures/4.png}
    \caption{\textit{Create .}}
        \end{center}
\label{gambar}
\end{figure}

\begin{figure}
\item[5] Setelah di klik Create,Lalu Klik From a File untuk mengambil data yg akan dimasukkan ke aplikasi.

    \begin{center}
\includegraphics[scale=0.2]{figures/5.png}
    \caption{\textit{{Proses Input Data Excel}l}}
        \end{center}
\label{gambar}
\end{figure}

\begin{figure}
\item[6] Masukan data excel yang akan digunakan untuk membuat aplikasi.

    \begin{center}
\includegraphics[scale=0.2]{figures/6.png}
    \caption{\textit{Input Data Excel.}}
        \end{center}
\label{gambar}
\end{figure}

\begin{figure}
\item[7] Lalu Muncul Tampilan seperti ini. Di Table Nama saya beri ACADEMIC.

    \begin{center}
\includegraphics[scale=0.2]{figures/7.png}
    \caption{\textit{Pengisian Nama Table}}
        \end{center}
\label{gambar}
\end{figure}

\begin{figure}
\item[8] Proses Pembuatan Aplikasi Berhasil.

    \begin{center}
\includegraphics[scale=0.2]{figures/8.png}
    \caption{\textit{Aplikasi Berhasil}}
        \end{center}
\label{gambar}
\end{figure}

\begin{figure}
\item[9] Lalu muncul tampilan seperti ini jika mengklik create aplikasi.

    \begin{center}
\includegraphics[scale=0.2]{figures/9.png}
    \caption{\textit{Tampilan Create Application}}
        \end{center}
\label{gambar}
\end{figure}

\begin{figure}
\item[10] Klik Run Application .

    \begin{center}
\includegraphics[scale=0.2]{figures/10.png}
    \caption{\textit{Create Application}}
        \end{center}
\label{gambar}
\end{figure}

\begin{figure}
\item[11] Lalu Masukkan username dan password oracle apex.
    \begin{center}
\includegraphics[scale=0.2]{figures/11.png}
    \caption{\textit{Run}}
        \end{center}
\label{gambar}
\end{figure}

\begin{figure}
\item[12] setelah itu muncul tampilan seperti ini.

    \begin{center}
\includegraphics[scale=0.2]{figures/12.png}
    \caption{\textit{Masukan Email}}
        \end{center}
\label{gambar}
\end{figure}

\begin{figure}
\item[13] Di gambar dibawah ini adalah untuk sign in ke oracle apex.

    \begin{center}
\includegraphics[scale=0.2]{figures/13.png}
    \caption{\textit{Tampilan Berhasil.}}
        \end{center}
\label{gambar}
\end{figure}


\begin{figure}

\item[14] Link Akses 
    \begin{center}
\includegraphics[scale=0.7]{figures/15.PNG}
    \caption{\textit{Tampilan Link.}}
        \end{center}
\item[15] username   : ariyoatmojo92@gmail.com
\item[16] kata sandi : seperti digambar berikut
    \begin{center}
\includegraphics[scale=0.9]{figures/14.png}
    \caption{\textit{Username}}
        \end{center}
\label{gambar}
\end{figure}

\end{enumerate}
