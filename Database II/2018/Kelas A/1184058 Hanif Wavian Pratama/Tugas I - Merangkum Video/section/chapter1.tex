\chapter{Oracle Aplication Express}

\section{Pengenalan Oracle APEX}
Oracle Aplication Express, adalah aplikasi yang digunakan untuk membangun database yang dikembangkan oleh Oracle. Selain itu, Application Express memungkinkan pengguna untuk merancang, mengembangkan, dan menggunakan aplikasi berbasis database yang baik dan responsif. Hanya menggunakan browser web, pengguna dapat mengembangkan dan menggunakan aplikasi profesional yang cepat dan aman untuk perangkat apa pun.

\section{Fitur-Fitur Pada Oracle Apex}
Berikut adalah fitur-fitur yang terdapat di aplikasi Oracle Experes :

\begin{enumerate}

\item[1]Drag and Drop file XLS, CSV, XML, atau JSON.
Jadi fitur bisa pengguna gunakan untuk mengembangkan aplikasi yang ingin dibuat dengan oracle apex dengan cara drag and drop file berupa XLS, CSV, XML, atau JSON.
    

\item[2]Membuat tabel dalam Autonomous Database. Oracle Autonomous Database mengotomatiskan semua manajemen database, infrastruktur, pemantauan, dan tuning. Hal ini dapat mengurangi biaya admin, meskipun admin masih akan diperlukan untuk tugas-tugas seperti mengelola bagaimana aplikasi terhubung ke gudang data dan bagaimana pengembang menggunakan fitur dan fungsi dalam basis data.


\item[3]Upload data into a new table. Fitur ini memungkinkan pengguna untuk mengunggah data ke table pada database yang sudah di buat di aplikasi apex.

\item[4]Create App based on new table. tidak hanya dapat menggugah data ataupun drag and drop file dari luar, dengan membuat table baru pada apex oracle, pengguna bisa membuat aplikasi berdasarkan tabel baru yang sudah dibuat oleh pengguna.\\

\end{enumerate}

\section{Cara Membuat Aplikasi Pada Oracle Apex}

\begin{enumerate}

\item[1] Buka website resmi apex.oracle.com dan siapkan email yang aktif untuk membuat workspace yang baru.
    
\item[2] Lalu, klik "Get start for free".

\item[3] Pilih Request a Free Workspace.

\item[4] Isi data diri seperti first name, last name, email dan nama workspace.

\item[5] Pilih yes/no pada pertanyaan "Are you new to Oracle Application Express" dan "Do you plan this workspace for a university class or training", setelah itu pilih next.

\item[6] Isi pada kolom yang terdapat pertanyaan "Why are you requesting this service ?", lalu klik next.

\item[7] Ceklis "I accept the terms", klik next.

\item[8] Setelah itu akan tampil halaman untuk mengkonfirmasi apakah ini anda, lalu klik submit request.

\item[9] Lalu anda akan menerima e-mail untuk mengkonfirmasi workspace yang sudah dibuat.

\item[10] Setelah mengkontifmasi di e-mail, lanjutkan ke sign in.

\item[11] Sign in ke akun oracle yang sudah dibuat.

\item[12] Anda akan masuk ke tampilan awal Oracle Express.

\item[13] Pilih "App Builder", lalu klik "Create New App".

\item[13] Pilih "Copy and Paste" pada select CSV, pilih project and tables, lalu klik next.

\item[14] Setelah load data, scroll ke bawah lalu setting table owner,table name error table name dan juga primary key.

\item[15] Load daat sukses, klik continue to create aplication wizard.

\item[16] Buat nama app from a spreadsheet dan pada features klik check all.

\item[17] Scroll ke bawah, pilih create application.

\item[18] Tunggu loading berberapa saat, lalu akan tampil halaman app builder project spreadsheet, klik run application.

\item[19] Login ke aplikasi dengan akun Oracle Expres yang tadi sudah dibuat.

\item[20] Aplikasi telah berjalan, selesai.
\end{enumerate}


