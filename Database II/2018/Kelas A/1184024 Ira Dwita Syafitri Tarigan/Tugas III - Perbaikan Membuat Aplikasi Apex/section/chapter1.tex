\chapter{Oracle Aplication Express}


\section{Pengertian Oracle APEX}
Oracle Aplication Express\cite{OracleApex}. Adalah sebuah wadah dan sarana untuk membuat aplikasi yang menggunakan database Oracle Itu sendiri, pada kelas Online pertama saya belajar banyak hal cara Menggunakan Aplikasi Oracle Apex online yang di dalam video sudah diberikan link diantaranya Request Workspace, Create Workspace, Membuat Spreadsheet Pertama.

\section{Cara Membuat Aplikasi Sederhana di Apex Oracle Online}
Langkah - langkahnya adalah sebagai berikut :
\begin{enumerate}
    \item Melakukan registrasi untuk masuk pada apex . di tampilan ini user memasukkan nama workspace yang telah dibuat serta memasukkan username dan password. ketika user belum mempunyai workspace . maka user mengklik request a workspace untuk membuat workspace baru .
\newline
    \item   Lalu request a workspace dengan mengisikan first name, last name
    , email dan membuat workspace . kemudian klik next 
\newline
    \item   verifikasi memlalui email . buka email lalu klik pesan dari oracle kemudian klik clik create workspace .
\newline
    \item   Masukkan oracle application express dengan memasukkan nama workspace , masukkan nama user dan password lalu klik sign in
\newline
    \item  Akan keluar tampilan seperti gambar yang sudah terlampir pada platform
    \newline
    \begin{figure} [!htbp]
        \centering
        \includegraphics[scale=0.5]{figures/1.png}
        \caption{Create}
        \label{penanda}
    \end{figure}
\newline
    \item  lalu create
\newline
    \item  Setelah itu unggah file dalam bentuk csv,xlxs,xml atau json .
    \newline
    \begin{figure}[!htbp]
        \centering
        \includegraphics[scale=0.5]{figures/3.png}
        \caption{Memasukan file}
        \label{penanda}
    \end{figure}
\newline
    \item  Kemudian pilih file data excel yang telah dibuat dalam bentuk xlsx
\newline
    \item  Buat tabel dengan memasukkan nama tabel disini saya membuat tabel Mahasiswa terlebih dahulu .
    \newline
    \begin{figure}[!htbp]
        \centering
        \includegraphics[scale=0.5]{figures/4.png}
        \caption{Membuat nama tabel Mahasiswa}
        \label{fig:my_label}
    \end{figure}
\newline
    \item Klik configure untuk mengecek apakah tabel dari data excel sudah benar atau tidak 
\newline
    \item  Lalu scroll kebawah sehingga keluar daftar nama mahasiswa yang berada pada file excel kemudian klik load data 
\newline
    \item Kemudia klik X untuk membuat tabel berikutnya , lakukan seperti perintah yang diatas lagi . selanjutnya membuat tabel dosen 
\newline
    \begin{figure}[!htbp]
        \centering
        \includegraphics[scale=0.5]{figures/7.png}
        \caption{Membuat nama tabel dosen}
        \label{fig:my_label}
    \end{figure}
\newline
    \item Lalu configure dan klik load data , lalu klik X untuk keluar .
    \newline
    \item Kemudian membuat tabel Kuliah .
    \begin{figure}[!htbp]
        \centering
        \includegraphics[scale=0.5]{figures/8.png}
        \caption{Membuat nama tabel kuliah}
        \label{fig:my_label}
    \end{figure}
    \newline
    \item Lakukan lagi perintah seperti diatas sampai semua tabel selesai dibuat . setelah tabel kuliah membuat tabel jadwal .
    \newline
    \begin{figure}[!htbp]
        \centering
        \includegraphics[scale=0.5]{figures/9.png}
        \caption{Membuat nama tabel jadwal}
        \label{fig:my_label}
    \end{figure}
    \newline
    \item Membuat tabel nilai . 
    \newline
    \begin{figure}[!htbp]
        \centering
        \includegraphics[scale=0.5]{figures/10.png}
        \caption{Mmebuat nama tabel nilai}
        \label{fig:my_label}
    \end{figure}
    \newline
    \item  Tampilan selanjutnya adalah create lalu hapus ID disetiap tabel .
\newline
\begin{figure}[!htbp]
    \centering
    \includegraphics[scale=0.5]{figures/12.png}
    \caption{Menghapus ID}
    \label{fig:my_label}
\end{figure}
\newline
\item Lalu ubah setaip tabel ke primary dan foreign key , pertama kita ubah terlebih dahulu yang primary seperti Dosen , Mahasiswa dan kuliah sedangan jadwal dan nilai kita ubah ke foreign key .
\newline
\begin{figure}[!htbp]
    \centering
    \includegraphics[scale=0.3]{figures/20.png}
    \caption{Mengubah tabel menjadi primary dan foreign key}
    \label{fig:my_label}
\end{figure}
\newline
\item Setelah itu kita buka app builder dan create .
\newline
\begin{figure}[!htbp]
    \centering
    \includegraphics[scale=0.5]{figures/24.png}
    \caption{App Builder}
    \label{fig:my_label}
\end{figure}
\newline
\item Kemudian add page untuk memilih fitur apa saja yang mau digunakan . 
\newline
\begin{figure}[!htbp]
    \centering
    \includegraphics[scale=0.5]{figures/26.png}
    \caption{Add page}
    \label{fig:my_label}
\end{figure}
\newline
\item Setelah itu Create An application 
\newline
\begin{figure}[!htbp]
    \centering
    \includegraphics[scale=0.5]{figures/38.png}
    \caption{Create An Application}
    \label{fig:my_label}
\end{figure}
\newline
    \item  lalu klik check all
\newline
    \item  Maka selanjunya jalankan aplikasi dengan mengklik run application
\newline
\begin{figure}[!htbp]
    \centering
    \includegraphics[scale=0.5]{figures/40.png}
    \caption{Run Application}
    \label{fig:my_label}
\end{figure}
\newline

\item Lalu masukkan email dan password untuk masuk ke aplikasi nya . 
\newline
\begin{figure}[!htbp]
    \centering
    \includegraphics[scale=0.7]{figures/41.png}
    \caption{Sign in}
    \label{fig:my_label}
\end{figure}
\newline
    \item  Maka akan muncul aplikasi sistem sederhana akademik 
\newline
\begin{figure}[!htbp]
    \centering
    \includegraphics[scale=0.3]{figures/42.png}
    \caption{Aplikasi Sistem Akademik Sederhana}
    \label{fig:my_label}
\end{figure}
\newline
    \item  Aplikasi akademik selesai dibuat .
\end{enumerate}

link : https://apex.oracle.com/pls/apex/f?p=76816:LOGIN_DESKTOP:711761223598218:::::

Email : IRADWITA22@GMAIL.COM

Password :#Ira1234






\begin{enumerate}


\end{enumerate}
