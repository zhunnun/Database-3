\chapter{Oracle Aplication Express}

\section{Oracle APEX}
Chaitanya Koratamaddi manajer produk utama.bekerja di Oracle, Bengaluru, India sejak 2005. Bertanggung jawab atas manajemen produk APEX sejak 2010 berbasis di Hyderabad, India.

\section{Pengertian Oracle APEX}
Oracle Aplication Express\cite{OracleApex}. Adalah sebuah wadah dan sarana untuk membuat aplikasi yang menggunakan database Oracle Itu sendiri, pada kelas Online pertama saya belajar banyak hal cara Menggunakan Aplikasi Oracle Apex online yang di dalam video sudah diberikan link diantaranya Request Workspace, Create Workspace, Membuat Spreadsheet Pertama.

\section{Pengembangan aplikasi pada enterprise}
1.	Membutuhkan sumber pengembangan khusus dan mahal
2.	Siklus pengembangan aplikasi yang panjang’’
3.	Jaminan simpanan yang besar
4.	Kolaborasi minimal
5.	Bisnis memecahkan masalah dengan tools yang salah

\section{Aplikasi yang aman untuk oracle apex}
1.	Aplikasi yang kritis untuk ribuan pengguna
2.	Mengisi kekosongan dalam perusahaan
3.	Merampingkan proses bisnis yang sudah ketinggalan zaman
4.	Memodrenkan sistem terdahulu
5.	Aplikasi self service untuk semua pegawai
6.	Customer/ partner facing portals
7.	Dll

\section{Tahapan Membuat Aplikasi Oracle Apex}
Pertama kita membuka website https://apex.oracle.com, disini kita akan mendapatkan akses untuk memasuki Oracle Apllication Express, siapkan email anda yang valid untuk membuat Workspace. 

1) Langkah 1

a. Buka Link https://apex.oracle.com

b.  klik memulai secara gratis 
\newline

2) klik permintaan ruang kerja gratis
\newline

3) login

a.  Masuk keruang kerja apex
\newline

4)  memilih jenis aplikasi
\newline

5)  Memuat data sampel
\newline

6)  Penamaan tabel
\newline

7)  Memverifikasi catatan dimuat
\newline

8)  Memberi nama aplikasi
\newline

9)  Membuat aplikasi 
\newline

10) Memberi nama aplikasi
\newline

11) Aplikasi dijalankan
\newline

12)  Mengurutkan Repositori Interaktif

a.	Klik spreadsheet

b.	Klik actions , pilih data , pilih sort

c.	Untuk 1 , pilih start date , yg kedua , pilih end daate lalu klik apply 
\newline

13) Tambahkan Perhitungan

a.	Klik actions , kemudian pilih data , lalu pilih compute

b.	Kemudian label kolom masuk budget v cost 

c.	Pilih format mask $5,234.10

d.	Masukkan computation expression I – H

e.	Setelah itu klik Apply
\newline

14) Tambahkan Grafik

a.	Klik actions lalu pilih chart 

b.	Pilih label project 

c.	 Pilih value-nya **Budget V Cost

d.	Untuk Function, pilih Sum

e.	Untuk Sort, pilih Label Ascending

f.	Plih Horizontal pada Orientation

g.	Setelah itu klik Apply
\newline

15) Save Report (Menyimpan Laporan)

a.	Klik Actions, setelah itu pilih Report, dan pilih Save Report

b.	Untuk menyimpan, pilih Report, setelah itu Save Report

c.	Pada Default Type, pilih Alternative

d.	Pada Name, masuk ke Data Review

e.	Klik Apply
\newline

16) Restrict The Status (A)

a.	Pada lingkungan runtime, klik ikon record  pada sebuah record

b.	Halaman modal akan ditampilkan

c.	Di Developer Toolbar, klik Quick Edit

d.	Arahkan kursor ke item Status (sampai garis biru menghilang) dan klik kiri.

e.	Page Designer tampil dengan memfokuskan Status Item
\newline

17) Restrict The Status (B)

a.	Pada Halaman Designer (Page Designer), dengan Property Editor (panel kanan), untuk   pilihan Type Select, pilih Select List

b.	Pada List of Values, for type select SQL Query

c.	Selanjutnya p pada SQL Query, klik Code Editor
\newline

18) Restict The Status (C)

a.	Dengan Code Editor, ikuti langkah-langkah berikut:
    Pilih distinct status d, status r dari spreadsheet, lakukan satu persatu

b.	Klik  Validate

c.	Klik OK

d.	Pada tampilan Extra values, pilih No

e.	Pada tampilan Null Value, masuk ke –Select Status-

f.	Klik Save (pada menu toolbar di atas kanan
\newline

19) Jalankan Aplikasi

a. Navigasikan kembali ke runtime environment

b. menyegarkan browser

c. mengedit record

d. klik status
\newline


\begin{enumerate}


\end{enumerate}
