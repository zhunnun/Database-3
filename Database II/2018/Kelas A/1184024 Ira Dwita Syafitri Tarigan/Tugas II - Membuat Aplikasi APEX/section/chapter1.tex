\chapter{Oracle Aplication Express}


\section{Pengertian Oracle APEX}
Oracle Aplication Express\cite{OracleApex}. Adalah sebuah wadah dan sarana untuk membuat aplikasi yang menggunakan database Oracle Itu sendiri, pada kelas Online pertama saya belajar banyak hal cara Menggunakan Aplikasi Oracle Apex online yang di dalam video sudah diberikan link diantaranya Request Workspace, Create Workspace, Membuat Spreadsheet Pertama.

\section{Cara Membuat App Builder Pada Oracle Application Express Online}
Langkah - langkahnya adalah sebagai berikut :
\begin{enumerate}
    \item Melakukan registrasi untuk masuk pada apex . di tampilan ini user memasukkan nama workspace yang telah dibuat serta memasukkan username dan password. ketika user belum mempunyai workspace . maka user mengklik request a workspace untuk membuat workspace baru .
\newline

    \item   Lalu request a workspace dengan mengisikan first name, last name
    , email dan membuat workspace . kemudian klik next 
\newline

    \item   verifikasi memlalui email . buka email lalu klik pesan dari oracle kemudian klik clik create workspace .
\newline

    \item   Masukkan oracle application express dengan memasukkan nama workspace , masukkan nama user dan password lalu klik sign in
\newline

    \item  Akan keluar tampilan seperti gambar yang sudah terlampir pada platform
\newline

    \item  lalu create
\newline

    \item  Setelah itu unggah file dalam bentuk csv,xlxs,xml atau json .
\newline

    \item  Kemudian pilih file data excel yang telah dibuat dalam bentuk xlsx
\newline

    \item  Buat tabel dengan memasukkan nama tabel 
\newline

    \item  Lalu scroll kebawah sehingga keluar daftar nama mahasiswa yang berada pada file excel kemudian klik load data 
\newline

    \item  Tampilan selanjutnya adalah create an application dengan create nama Akademik .
\newline

    \item  lalu klik check all
\newline

    \item  Maka selanjunya jalankan aplikasi dengan mengklik Eun application
\newline

    \item  Aplikasi akademik akan mencul 
\newline

    \item  Kemudia masukkan user email kita dan password yang telah dibuat tadi . lalu klik sign in .
\newline

    \item  Aplikasi akademik selesai dibuat .
\newline
\end{enumerate}

link : https://apex.oracle.com/pls/apex/f?p=78901:LOGIN_DESKTOP:707245363740804:::::

Email : IRADWITA22@GMAIL.COM

Password :#Ira1234






\section






\begin{enumerate}


\end{enumerate}
